%Husk at dette kun er en beskrivelse!

%  Hvordan forbereder I møder med jeres vejleder ?

%  Hvilken type respons ønsker I fra vejlederen ?

%  Har I en samarbejdsaftale med jeres vejleder ?

%  Hvilken type vejledning har I modtaget ? Var det hvad I ønskede jer ?

%  Hvilke eksperimenter har I gennemført for at forbedre samarbejdet med jeres vejledere ?

%  Hvad er jeres erfaringer fra de eksperimenter ?

%  Hvilke forventninger vil I stille til jeres vejledere fremover ?

\textbf{Samarbejdsaftale}\\ % Jimmi
I forbindelse med første vejledermøde, fik gruppen snakket med vejlederen, om hans forventninger og krav til projektudarbejdelsen. Ligeledes havde gruppen nogle forventninger til vejlederen, som blev diskuteret. Dette førte til vejleder-samarbejdsaftalen, hvoraf følgende kan nævnes af vejlederens krav:
\begin{itemize}
    \item Deadline for indsendelse af materiale til møder er senest 24 timer og skal foreligge på den forekommende hverdag
    \item Følgende information skal følge med e-mails til vejlederen
    \begin{itemize}
        \item Dokument-headeren skal indeholde følgende: Filnavn, år-måned-dag, gruppe id
    \end{itemize}
\end{itemize}
Gruppen følte at det var vigtigt at få en samarbejdsaftale på plads, for at få nogle klare retningslinjer på plads. Ligeledes medvirkede samarbejdsaftalen til en række diskussioner, som var gode at have tidligt i processen. Se \ref{bil:vejleder_samarbejdsaftale} for hele vejleder-samarbejdsaftalen

\textbf{Vejledermøderne}\\ % Jimmi
Inden vejledermøderne sendte gruppen en dagsorden til vejlederen. Der var i henhold til vejleder-samarbejdsaftalen opstillet en 24-timers deadline, som gruppen bestræb sig på at efterleve. Denne deadline blev dog uheldigvis overskredet et par gange, og én enkelt gang fik gruppen ikke tilsendt noget materiale til forinden vejledermødet. \\

Vejledermøderne forløb godt, og vejlederen var god til at komme med konstruktivt feedback på det tilsendte materiale og de yderligere spørgsmål. Vejlederen tog endvidere intuitiv til at forklare emner, som gruppen havde manglede forståelse for – her kan især de forskellige udviklingsmetoder berettiges. \\

Størstedelen af vejledermøderne var tidslagt kl. 09.00, hvilket gav gruppen mulighed for at køre en kort briefing på projektets status. Dette medvirkede til at gruppens medlemmer følte sig godt forberedte til vejledermøderne.
