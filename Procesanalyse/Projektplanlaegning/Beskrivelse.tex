\subsection{Beskrivelse}
%Husk at dette kun er en beskrivelse!

\textbf{Planlægning og koordinering}\\ %Martin
%Trello, tavlen, tid
På baggrund af erfaringer fra P1 valgte gruppen at bruge det organisatoriske værktøj Trello, som tager form af et online-tilgængeligt antal "opslagstavler", hvori der kan oprettes søjler, som kan indeholde et antal stærkt redigerbare "opslag". Der blev oprettet to opslagstavler, til henholdsvis program og rapport, og disse blev udfyldt med søjler som "Problembeskrivelse" og "Problemløsning", hvori opgaver relaterende til de forskellige afsnit blev oprettet, resulterende i at et overblik over den nuværende status, samt gruppemedlemmers umiddelbare opgaver, altid kunne tilgåes, uanset tid og fysisk placering.\\
For at få et mere håndgribeligt overblik efter en begivenhed der skabte mange nye opgaver, såsom en vejledersamtale eller en iteration med kunden, blev tavlen i grupperummet i visse tilfælde benyttet. De nyfundne opgaver blev da skrevet op på tavlen, og efterhånden hvisket ud som de blev løst.\\
For at holde overblik over deadlines blev Trello ligeledes benyttet. Trello implementerer en intergreret kalender, så et overblik over kommende deadlines let bliver skabt. Ydermere blev en søjle ved navn "Tidsplan" oprettet i projekt opslagstavlen, hvori de mest overordnede deadlines blev indsat, således at der altid blev kastet et blik på de nærmeste deadlines, når et gruppemedlem skulle ind at vælge nye opgaver eller slette færdiggjorte.

% I analyse: Motivationsfaktor: De dage hvor vi brugte tavlen i grupperummet var ofte de mest produktive.
% I analyse: Vi er pisseringe til at lægge en tidsplan. Analyser på det.

% Søren
\textbf{Versionskontrol}\\ %Søren
%TFS
Til at holde styr på kodebasen til projektet blev der benyttet en TFS server\footnote{TFS: Team Foundation server}.\\

\textbf{Rapport}\\ %Søren
%Sharelatex
Rapportskrivningen er i gruppen prinært fortaget vha. ShareLatex - en online LaTeX editor. I starten skete det gennem den officielle version af ShareLatex, men efter at ShareLatex blev open source opsatte gruppen sin egen server.\\

\textbf{Fildeling}\\ %Søren
%Google Drive
Til deling af filer og skrivning af referat og dagsorden er Google Drive blevet benyttet. I Google Drive er der fra start blevet lagt op til at der skal være orden så det er let at lokalisere en given fil, eksempelvis referater.