\subsection{Analyse}

\textbf{Planlægning og koordinering}\\ % Martin
%Trello, tavlen, tid
Der er store fordele ved at have et lettilgængeligt værktøj til at organisere arbejdsindsatsen. Gruppen havde en periode hvor ingen tog initiativ til at opdatere Trello, hvilket resulterede i at gruppen manglede overblik og retning. Humøret blev ringere, da det ikke var muligt at se fremgang, selvom der blev puttet mange timers arbejde ind. Det er blevet tydeligt for gruppen, at der findes en stærk motivationsfaktor i hele tiden at vide præcis hvor man er i arbejdsprocessen, hvilke opgaver der er blevet færdiggjort og hvilke opgaver der kommer efterfølgende. Alene handlingen at fjerne en opgave fra tavlen, så den samlede mængde af nuværende opgaver er blevet formindsket og gruppen er lidt tættere på at være færdig, føles som en stor motivationsfaktor, fordi man kan se en håndgribelig fremgang. Denne motivationsfaktor bidrager også til at nogle af de ubestrideligt mest produktive dage i grupperummet, var når vi\mafix{Er "vi" okay?? Undskyld, jeg er et vrag.} skrev dagens opgaver op på tavler, og derefter brugte resten af dagen på at strege dem ud.\\\\
En ulempe ved et sådant system, er at gruppens medlemmer skal gide bruge det. Det er meget arbejde at sætte sig ind i præcis hvad status er på projektet og opdate tavlen, især hvis folk aligevel ikke gider at bruge systemet. Hvis medlemmerne ikke gider at sætte deres navn på nogle opgaver og fjerne opgaver på tavlen når de er færdige, og det står til en eller to mand hele tiden at opdatere tavlen, så bliver systemet ineffektivt og mere besvær end det er værd. Dette problem har gruppen til dels haft, resulterende i korte perioder med inaktivitet, hvorefter gruppen tog sig selv i nakken og begyndte at bruge Trello igen.


%På baggrund af erfaringer fra P1 valgte gruppen at bruge det organisatoriske værktøj Trello, som tager form af et online-tilgængeligt antal "opslagstavler", hvori der kan oprettes søjler, som kan indeholde et antal stærkt redigerbare "opslag". Der blev oprettet to opslagstavler, til henholdsvis program og rapport, og disse blev udfyldt med søjler som "Problembeskrivelse" og "Problemløsning", hvori opgaver relaterende til de forskellige afsnit blev oprettet, resulterende i at et overblik over den nuværende status, samt gruppemedlemmers umiddelbare opgaver, altid kunne tilgåes, uanset tid og fysisk placering.\\
%For at få et mere håndgribeligt overblik efter en begivenhed der skabte mange nye opgaver, såsom en vejledersamtale eller en iteration med kunden, blev tavlen i grupperummet i visse tilfælde benyttet. De nyfundne opgaver blev da skrevet op på tavlen, og efterhånden hvisket ud som de blev løst.\\
%For at holde overblik over deadlines blev Trello ligeledes benyttet. Trello implementerer en intergreret kalender, så et overblik over kommende deadlines let bliver skabt. Ydermere blev en søjle ved navn "Tidsplan" oprettet i projekt opslagstavlen, hvori de mest overordnede deadlines blev indsat, således at der altid blev kastet et blik på de nærmeste deadlines, når et gruppemedlem skulle ind at vælge nye opgaver eller slette færdiggjorte.


\textbf{Versionskontrol}\\  % Søren
%TFS
Versionskontrol gjorde det lettere at arbejde flere personer på projektet, idet at hele kodebasen var på en server. Under projektperioden blev der lavet ca. 1100 commits.\\

Gruppen valgte at benytte TFS pga. den gennemarbejdede integration med Visual Studio. Serveren blev sat op af gruppen selv, men build serveren kom aldrig op og køre pga. manglende priotering på denne opgave. Build serveren kunne have gjort det umuligt at lave et checkin, hvis build fejlede. Dette ville have forhindret at der blev comittet kode der ikke kunne builde, hvilket skabte stor irritation for de personer der ellers arbejdede på systemet.\\

TFS gjorde det let at håndtere eventuelle konflikter mellem changesets med automatisk merge og vha. en merge editor direkte indbygget i den benyttede IDE, Visual Studio.\\

Der var dog til tider til stor irritationsmomment for gruppens medlemmer at skulle merge manuelt og der er eksempler på tilfælde, hvor noget der tidligere har virket har holdt op med at virke pga. fejl af personen der har merged ændringerne.\\


\textbf{Rapport}\\ %Søren
%Sharelatex
Selve opsætningen af LaTeX var meget lettere dette semester pga. gruppen valgte at benytte P1 gruppen B2-24's preamble, som nogle af gruppemedlemmerne havde været med til at oprette.\\

Det var dog ikke alle af gruppens medlemmer der benyttede ShareLaTeX pga. utilfredshed med teksteditoren i ShareLaTeX.\\

Udover dette var ShareLaTeX også til tider ustabilt specielt tæt på afslutningen af projektet. Dette skyldes sandsynligvis ustabalitet i den udgave af ShareLaTeX vi havde installeret, som var en af de første udgivne open source udgaver af ShareLaTeX. Det blev vurderet til at være for risikabelt at forsøge at opdatere ShareLaTeX før afsluttelsen af projektet.\\

\textbf{Fildeling}\\ %Søren
%Google Drive