\section{Forventninger til P2}

Gruppen er sammensat af fem medlemmer fra samme P1 gruppe samt to medlemmer fra to forskellige P1 grupper. Dette bevirkede, at de fleste af gruppemedlemmerne kendte hinandens kompetenceniveau og arbejdsform fra starten af, hvilket resulterede i, at gruppen var sikker i bedømmelsen af den håndterbare potentielle arbejdsmængde, da denne blev sat højt. P1 gruppen, som størstedelen af medlemmerne kom fra, havde ydermere været relativt ambitiøse i den arbejdsmængde, som de havde tillagt sig selv, og følte, at de havde fået et væssentligt udbytte i ekspertise og erfaring under denne fremgangsmåde. Dette bevirkede, at gruppen ønskede at presse sig selv, og håbede at stå tilbage efter endt arbejde med en rapport og et produkt af høj kvalitet og en tilsvarende middel-høj karakter. Dette afspejledes også i den relativt høje tidsmængde, som gruppen var forberedt på at lægge i arbejdet, hvilket også er beskrevet i gruppekontrakten. En god antydning af gruppens ambitionsniveau var, at gruppen håbede at levere et færdigt system til kunden, Aalborg Roklub, ved slutningen af forløbet.