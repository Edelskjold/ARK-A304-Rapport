%Vedhæft samarbejdsaftalen
%Husk at dette kun er en beskrivelse!

%  Har I talt om jeres forventninger til hinanden ? (F.eks. om hvad der motiverer jer, ambitionsniveau, social deltagelse ???)

%  Hvordan afvikler I møder ? (mødeleder ? - runde om bordet ? Andet ?)

%  Hvordan er kommunikationen i jeres gruppe ? Er der nogle der taler hele tiden ? Er der nogen der aldrig siger noget ? Bruger gruppen uforholdsvis lang tid på diskussionerne ? Hvorfor ?

%  Hvordan har I arbejdet med at motivere hinanden i gruppen ?

%  Hvordan har I forsøgt at anvende viden om jeres konfliktadfærd til at udvikle gruppen konstruktivt ?

%  Hvordan er jeres konfliktberedskab ?

%  Hvad har I gjort for at udvikle jeres samarbejde i en positiv retning ?

%  Hvilke eksperimenter har I gennemført i den forbindelse ?

%  Hvilke forventninger har I til samarbejdet fremover ? Hvordan skal de blive opfyldt ?

\textbf{Arbejdsfordeling}\\ % Rune
%Kompetencer (-samarbejdet internt)
I forbindelse med fordelingen af arbejdskraft på forskellige områder, valgte gruppen ikke at tage hensyn til medlemmernes kompetencer, idet at det ansås som en vigtigtighed, at alle gruppens medlemmer kom omkring alle aspekter af projektarbejdet, så meget som det var muligt i forhold til tidsbegrænsning. Tanken bag dette var at forøge forståelsen for alle dele af rapporten og produktet for hvert medlem, samt øge læring indenfor de områder, som et medlem ikke normalt bevægede sig ud i. For at maksimere muligheden for vækst blev der hovedsageligt arbejdet i grupperummene, især i programmeringssammenhæng, således at de teknisk svagere medlemmer altid havde et teknisk stærkere medlem at spørge til råds.\\
Gruppen har opfyldt kravet om at stille en formand til rådighed for gruppestyremøderne, men dette har udelukkende været af formelle grunde, eftersom gruppen ikke ønsker at tildelde en leder-lignende rolle som formand, ordstyrer el.l. Der har i sin almindelighed været forskellige medlemmer fra gruppen, som i en naturlig tilgang har taget styringen - nogle mere end andre. 

\textbf{Samarbejdsaftale}\\ % Rune
%Kom ind på nogle få eksempler fra samarbejdsaftalen (de væsentlige tiltag)
Eftersom forrige P1 projekt havde været en success, så blev samarbejdsaftalen tilhørende forrige projekt gennemarbejdet af gruppen som et af de første opgaver i P2-forløbet. Alle medlemmer har haft mulighed for at komme med tiltag eller ændringer til samarbejdsaftalen, og den har ligeledes taget form efter gruppens ønsker og ambitionsniveau. Blandt punkterne indenfor samarbejdstalen er gruppen f.eks. blevet enige om:
\begin{enumerate}
    \item Vi følger de officielle mødetider som bruges til kurser, med mindre andet er aftalt. (8.15 - 16.15)
    \item Der afholdes 2 gruppemøder om ugen; henholdsvis Mandag og Torsdag eller Fredag hvis det kan lade sig gøre. Gruppemødet vil blive tidslagt udenfor forelæsninger.
    \item Det forventes at alle arbejder seriøst på projektet og lærer de nødvendige redskaber, såsom C#-programmering og LaTeX. Minimumskarakter som sigtes efter er 7.
    \item Det forventes at folk er parate til at arbejde i weekenderne og om aftenen - dette aftales på forhånd. Den ekstra indsats forventes i perioder; hvor det er påkrævet for at gøre emner færdige. Vi vil vha. Trello se emner som kræver en ekstra indsats. Dette vil ydermere blive diskuteret på gruppemøderne. 
    \item Hvis konflikter skulle være under optrapning, skal der tages en fælles snak i gruppen for hvordan disse konflikter kan blive løst hurtigst muligt.
\end{enumerate}

\textbf{Gruppemøder}\\ % Rune
%Hvordan fungerede gruppermøderne. Hvad var formålet?
Som udgangspunkt fungerede gruppemøderne som opsamlingspunkter, hvor projektet standpunkt kunne vurderes, og hvorfra gruppens medlemmer kunne få et overblik over projektet. Derudover har gruppemøderne været brugt til at diskutere vigtige beslutninger, der har haft betydning på længere sigt.
