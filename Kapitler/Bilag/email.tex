\label{bil:email}
I projektet har det været yderst essentielt at få svar på givende spørgsmål, som er opstået under udviklingsforløbet.
Hertil har gruppen haft en del email korrespondancer, hvor et udpluk af de vigtigeste er medtaget herunder\\
\\
{\textbf{Email-korrespondance (Spørgsmål) fra gruppen}}\\
Hej Karsten og Jens,\\
\\
Vi har lige et par spørgsmål til programmet.\\
\\
Skal styremanden på en tur, tælle med i kilometer-statistikkerne. Både de overordnede kilometre, såvel som de individuelle kilometre?\\
\\
I jeres båd-XML: \\
- I har en property som hedder 'Aktiv' - hvad bruger i denne til?\\
- Hvad bruger i jeres 'bådtype'-property til?\\
\\
Har i mulighed for at vi kan sætte MySQL op på jeres lokale server, således at man også kan tilgå daten udefra?\\

-- \\
Med venlig hilsen\\
Jimmi Nielsen\\
Gruppe: SW2A304, AAU, Softwareingeniør\\
Lokale: A304 \\
Mail: jiniel13@student.aau.dk\\

{\textbf{Email-korrespondance (svar) fa Aalborg Roklub}}\\
Hej Jimmi\\

{\textbf{Styrmænd:}}\\
Styrmænd tæller med på lige fod med alle andre i båden.\\
Specielt i inriggerne byttes der rundt under ro turen, så man kan ikke sådan pille dem ud \\

{\textbf{Båd-XML}}\\
Aktiv:
\begin{enumerate_small}
    \item Båden er i Aktiv tjeneste og kan anvendes
    \item Båden findes ikke i klubben længere – kan ikke anvendes
\end{enumerate_small}

{\textbf{BådType:}
\begin{enumerate_small}
    \item Robåd
    \item Kajak
    \item Ergometer
\end{enumerate_small}
På nuværende systems forside kan man vælge at filtrere på de tre typer.\\

{\textbf{MySQL}}\\
Vi har i dag 5 databaser hos surftown – det bruges af de forskellige systemer vi anvender på web delen.
Vi anvender sådan set alle fem databaser. Der er dog en ’sandbox’, som bruges til at afprøve større ændringer i vores wordpress.\\

Hvis det kun er til noget test, kan i sagtens låne den.\\
Når det bliver permanent, må vi finde ud af at rykker de andre systemer sammen, så der bliver plads til jeres database.
Det burde kunne lade sig gøre.\\
https://phpmyadmin.surftown.com/index.php\\
user: \hl{(Slettet, red.)}\\
pwd : \hl{(Slettet, red.)}\\
servervalg: \hl{(Slettet, red.)}\\
\\
jeg har mulighed for at give adgang ekstern databaseadgang. Jeg skal blot registrere den IP adresse der skal have adgang.\\
 
Mvh. Jens Brandt


