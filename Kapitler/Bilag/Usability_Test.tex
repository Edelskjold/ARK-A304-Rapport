\label{bil:usability}

\textbf{1. Testcase (Start rotur)}: Du skal starte en ny rotur, med nedenstående oplysninger:\\

Båden du ønsker at leje er: Laubek\\ 
Tilføj en besætning bestående af: 1 styrmand, 2 medlemmer, 2 gæster.\\
Vælg en retning: Øst\\

\textbf{2. Testcase (Afslut rotur)}: Du skal afslutte en rotur, som du lige er kommet hjem fra:\\

Båden du har benyttet er: Laubek\\
Vælg Standard tur: Nordbroen\\
Ændre distancen til 12 km\\

2.1 Testcase (ændre distance): Du har tastet en forkert distance, du ønsker derfor at ændre distancen til 15 km. Gør dette\\

\textbf{3. Testcase (Både på vandet)}: Mørket melder sig på, du ønsker at undersøge hvilke både der ikke er kommet hjem. Gør dette.\\

3.1 Testcase (besætning): Find ud af hvilken besætning den første båd i listen har.\\
3.2 Testcase (telefon-nummer): Find telefonnummeret på det første besætningsmedlem i båden.\\


\textbf{4. Testcase (Opret skade)}: Du er lige kommet hjem fra en rotur i båden Laubek. Du kom uheldigvis til at beskadige båden, så den ikke længere er anvendelig.\\

Skadetype: Skroget er blevet skævt.\\

\textbf{5. Testcase (Opret langtur)}: Du ønsker at tage på en langtur, og vil derfor gerne anmode om dette:\\

Til turen, ønsker du at reservere båden: Kap 71\\

Du ved på nuværende tidspunkt ikke hvilke besætningsmedlemmer der skal med på langturen, men vil stadig gerne anmode om langturen. Udfyld besætning.\\

Du ønsker at tage afsted d. 21-05-2014\\
Du forventer at komme tilbage d. 28-05-2014\\
Du forventer at overnatte i Nibe og Løgstør\\
Dine forventede dagsdistancer er 50 km\\
Din beskrivelse af langturen skal være: Hyggetur\\
Du skal vælge dig selv som ansvarlig