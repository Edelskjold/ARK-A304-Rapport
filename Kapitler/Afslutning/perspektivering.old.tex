% Mads
% Redigeret af Martin. 16-05-14.

\chapter{Perspektivering}
\label{sec:Perspektivering}
I dette afsnit vil der blive kigget meget bredere på hvad mulighederne for programmet (og projektet generelt) kunne være. Dette kan betragtes som en form for antagelser, hvor der ikke er vedlagt dokumentation, da der kigges meget bredt og generelt.\\

\textbf{Udlejning af både}\\
Da Aalborg roklub er omgivet af forskellige roklubber og sejlklubber, kunne man evt. indlede et samarbejde. Samarbejdet skulle gå ud på at roklubber på tværs af hinanden kunne låne materialer og arealer. Her kunne det være ideelt hvis systemet kunne booke f.eks. en båd på en bestemt tidspunkt, hvis ens egen klub ikke har en til rådighed. Dette ville på længere sigt give mulighed for at klubbens bestyrelse, ville kunne indlede aftaler med roklubber som er placeret i andre byer end Aalborg. Derved ville medlemmer af ordningen kunne låne en båd, hos en foreningen som ligger langt fra deres egen roklub, og derved komme ud i nye naturomgivelser. Det ville være sådan, at foreningen ville kunne vælge hvilke både, som de gerne ville låne ud til andre roklubber. Hertil ville det også være smart, hvis man kunne udleje lokaler til andre roklubber eller konfimationer som Aalborg roklub pt. gør manuelt.\\

\textbf{Online system}\\
Systemet kører pt. på foreningens computer, hvor det opfylder de givne krav, og nødvendigheder som Aalborg roklub har brug for. Derfor er der ikke et konkret behov for at kunne tilgå systemet online, dette ville være et behov som ville opstå hvis man skulle udleje både, til andre foreninger. Derfor kunne det være smart at systemet ville kunne tilgåes online, enten igennem foreningernes hjemmeside, som en form for plugin, eller at hele programmet ligger online. Dette giver derimode nogle begrænsninger, i forhold til at systemet derved hele tiden er afhængig af internettet, hvilket ikke anses for at være den store begrænsning nu til dags. Derfor ville man godt kunne lave programmet online, hvilket ville gøre det lettere hvis man skulle distribuere programmet til flere foreninger. Dette vil sige at man ville kunne udforme systemets funktioner i et CMS\footnote{Content Management System}, som derved ville give foreningen adgang til selv at opdatere deres program design til en hvis grad, såsom at skifte logo, og baggrundsfarve samt skrifttype. Ved at bruge en online løsning til at have programmet kørende på, ville det ligeledes gøre det billigere og mere attraktivt for foreningerne, da de ikke selv skal have elektronik kørende alle døgnets timer i deres foreningen og sørge for at vedligeholde dette.\\

\textbf{Medlems- og faktureringssystem}\\
Som beskrevet kort i diskussionen, er det nuværende program ufuldstændigt, da der er blevet afgrænset fra at inkludere et medlems- og faktureringssystem. For at gøre programmet mere attraktivt kunne disse meget centrale dele for enhver klubadministration inkluderes i løsningen. Det vil ofte være en fordel at have et enkelt program til at administrere en forening, da det nedsætter tidsforbrug og kompleksitetsniveau for den hovedsageligt frivilige arbejdskraft, som ofte driver sådanne foreninger. At lave et samlet system ville ligeledes nedsætte mængden af modifikationer, som ville være nødvendige, for at oversætte systemet fra en roklub til en anden. For eksempel, er det nuværende system bygget til at modtage regelmæssig data fra en XML-database, fordi Aalborg Roklubs medlemskartotek benytter dette filformat. En anden forening kunne benytte et andet filformat, som programmet derved skulle omskrives til at være kombatibel med, for at være brugbart for dem. Ud over dette, er der også visse problemer associeret med at hente data fra en seperat database, såsom valg af opdateringstidspunkter og inkombatibilitet mellem databasen og system, hvis der bliver ændret i dataformateringen. Med et samlet system skulle dataen kun føres over en gang, hvorved disse potentielle problemer ville kunne undgåes. \\

\textbf{Andre Roklubber}\\
Da programmet er udarbejdet efter en case (Aalborg Roklub) kunne det tænkes, at andre roklubber ligeledes kunne samme problemstillinger som er løst med programmet. Derfor kunne man forstille sig, at programmet kunne blive distribueret til andre roklubber, med deres identitet istedet for Aalborg roklubs. Hertil ville det være ligegyldigt om det er online baseret eller kører direkte på en computer, da det stadig ville kunne bruges, og at behovet kun opstår hvis det bliver aktuelt at udleje både fra foreningen.\\

Da gruppen havde første møde med Aalborg roklub, blev det udtalt at flere af disse ting, som Aalborg roklub gjorde manuelt (langtudsblanketter, skadesrapport osv.) gjorde de andre roklubber også. Derfor mener Aalborg roklub ligeledes, at programmet kunne være brugbart hos de andre klubber, med nogle ændringer i forhold til deres krav.\\

Programmets fremtid ser derfor meget lys ud, da muligheden for at kunne udvide det til andre foreninger er stort.\\

\textbf{Udlejning af både}\\
I projektoplæget står der beskrevet at emnet omhandler udlejning af både til undervisningsbrug, hvilket gruppen har valgt at vige væk fra. Gruppen har istedet valgt en case, med Aalborg roklub, hvorefter der er blevet taget udgangspunkt i deres problemstillinger. Emnet er tilnærmelsesvis indenfor kategorien, da det stadig omhandler både, dog ikke udlejning heraf. Derfor mener gruppen i samarbejde med vejlederen at projektoplægget er retvisende og der derfor godt kan tages afvigelser.\\