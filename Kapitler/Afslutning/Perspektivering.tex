% Mads
% Redigeret af Martin. 16-05-14.
%Jimmi 20-05-2014 1658
\chapter{Perspektivering}
\label{sec:Perspektivering}
I dette afsnit vil der blive undersøgt meget bredere, hvad mulighederne for programmet (og projektet generelt) kunne være. Dette kan betragtes som en form for antagelser, hvor der ikke er vedlagt dokumentation, da grundlaget til perspektiveringen er dannet gennem en abstraktion af det endelige programs potentielle udvikling. \\

\textbf{Udlejning af både}\\
Hvis Aalborg Roklub var omgivet af forskellige ro- og sejlklubber, kunne det være nemt at indlede et samarbejde på tværs af klubberne. Samarbejdet skulle gå ud på, at roklubber kunne udlåne materiel og lokaler til hinandens medlemmer. Her kunne det være ideelt, hvis systemet f.eks. kunne booke en båd til et medlem, der ikke er af samme klub, som båden tilhører. Dette koncept kunne på længere sigt også spredes til klubber i andre byer. Derved ville medlemmer af systemet også kunne låne en båd hos en foreningen, som ligger langt fra deres egen roklub. Det kunne gøres på en effektiv måde, således at klubberne selv kan afgøre, hvilke materialer de ønsker at stille til rådighed for andre klubber. \\

\textbf{Online system}\\
Systemet kører pt. på foreningens computer, hvor det opfylder de givne krav og nødvendigheder, som Aalborg Roklub har brug for. I forhold til protokolsystemet, traf Aalborg Roklub den beslutning, at systemet ikke bør kunne tilgås hjemmefra af medlemmerne. Online tilgængelighed kunne sagtens tiltænkes at være et ønske, som andre klubber har, hvilket ville kræve en omstrukturering af protokolsystemet. Der blev i samråd med Aalborg Roklub lagt vægt på, at materialeregistrering skulle foregå internt i roklubben på en touchskærm. Skulle protokolsystemet fungere online, så ville det være mest hensigtsmæssigt at konstruere en webbaseret løsning. Dette ville give medlemmerne mulighed for at registrere materiale hjemmefra. \\

Administrationssystemet blev konstrueret med fokus på online tilgængelighed. Dette var et valg, som blev truffet i samråd med Aalborg Roklub, der havde ønsket muligheden for at kunne administrere blanketternes status hjemmefra. Dette giver derimod nogle begrænsninger i forhold til, at systemet hele tiden er afhængigt af internettet, hvilket ikke anses for at være den store begrænsning nu til dags. \\

\textbf{Medlemssystem}\\
Som beskrevet kort i diskussionen, er gruppens program ikke fuldendt. Der blev foretaget en ekskludering af medlemskartoteket, som blev truffet på baggrund af, at Aalborg Roklubs eksisterende medlemskartotek var tilfredsstillende. Dette kunne give komplikationer i forhold til andre klubber, som nødvendigvis ikke har et tilsvarende system. For at gøre programmet mere attraktivt, kunne disse meget centrale dele for enhver klubadministration, inkluderes i løsningen. Det er dog til stadighed en vis fordel i at have et enkelt system, idet at kompleksitetsniveauet kan stige i takt med flere funktionaliteter. At lave et samlet system ville ligeledes nedsætte mængden af modifikationer, som ville være nødvendige, for at oversætte systemet fra en roklub til en anden. F.eks. er det nuværende system bygget til at modtage regelmæssig data fra en XML-database, fordi Aalborg Roklubs medlemskartotek benytter dette filformat. En anden forening kunne benytte et andet filformat, som programmet derved skulle omskrives til at være kompatibel med. Ud over dette, er der også visse problemer associeret med at hente data fra en særskilt database, såsom valg af opdateringstidspunkter og inkompatibilitet mellem databasen og system. Med et samlet system skulle daten kun føres over én gang, hvorved disse potentielle problemer ville undgås. \\

\textbf{Udlejning af både}\\
I projektoplægget står der beskrevet, at emnet omhandler udlejning af både til undervisningsbrug, hvilket gruppen ikke har implementeret. Dette valg blev truffet på baggrund af Aalborg Roklubs ønske om ikke at kunne reservere materialet fremadrettet. Dette er en funktion som sagtens kunne implementeres, såfremt andre roklubber ville have dette behov. \\

Gruppen har istedet valgt et case med Aalborg roklub, hvorefter der er blevet taget udgangspunkt i deres problemstillinger. Emnet er tilnærmelsesvist indenfor kategorien, da det stadig omhandler både. Derfor mener gruppen, i samarbejde med vejlederen, at projektoplægget er opfyldt.
