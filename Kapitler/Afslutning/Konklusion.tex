%Jimmi 20-05-2014 1513
% Rune 20-05-2014 16:45

\chapter{Konklusion}
\label{sec:konklusion}
Efter at have endt problemløsning, kan der nu konkluderes på, hvor godt gruppens færdige program løser problemformuleringen, og opfylder kravspecifikationens opstillede krav. Gruppens løsning har resulteret i to seperate programmer. Eftersom disse programmer er stillet vidt forskellige krav, så ses det som en nødvendighed at evaluere dem separat. \\

I forbindelse med administrationssystemet blev der lagt vægt på online tilgængelighed, tidsbesparelse, overblik og sikkerhed. Da der ikke er blevet foretaget en test med Aalborg Roklubs bestyrelse, kan der ikke drages nogen konklusioner, om hvorvidt de finder løsningen tilfredsstillende. Der kan dog evalueres på de nye funktionaliteter, som er blevet implementeret i administrationssystemet. \\

I forhold til forøget overblik, online tilgængelighed og tidsbesparelse, så bidrager flere af programmets funktioner til disse. Digitaliseringen af langturs- og skadeblanketter, samt det forbedrede statussystem, hvor man kan se en båds tilstand, ture og benyttende medlemmer, skaber et overblik over klubbens ejendele, hvilket ikke var muligt med det gamle system. \\

Disse funktioner fører ligeledes til en tidsbesparelse, idet systemet udfører visse automatiske handlinger, såsom at notificere involverede parter om, at en langtursblanket er blevet behandlet, eller at gøre en båd inaktiv, hvis en skadesanmeldelse foreslår dette. En væsentlig tidsbesparelse i sig selv er også det faktum, at der er online tilgængelighed, hvilket bevirker, at klubbens administration ikke er bundet af en fysisk tilstedeværelse i klubbens lokaler.\\

I forbindelse med øget sikkerhed er der blevet konstrueret et SMS-system, som sender en bekræftelses-SMS til medlemmer hvis de ikke er kommet tilbage inden mørkets frembrud. Hvis medlemmerne ikke svarer indenfor en given tidsperiode, notificeres udvalgte bestyrelsesmedlemmer. Med disse funktioners implementering, betragtes den del af problemformuleringen, der henvender sig til administrationssystemet, som besvaret fyldestgørende. \\

Under konstruktionen af protokolsystemet blev der hovedsageligt fokuseret på brugervenligheden, da det gennemsnitlige it-kyndighedsniveau i Aalborg Roklub kan antages at være lavt. For at undersøge i hvor høj en grad ønsket om et brugervenligt protokolsystem blev opfyldt, og hvorvidt produktet kunne forbedres, blev en usabilitytest foretaget blandt Aalborg Roklubs medlemmer, som kan ses i afsnit \ref{subsec:usability_ark}. Denne usabilitytest viste, at systemet stadig mangler noget arbejde for at nå op på et niveau, hvor problemformuleringen kan siges at være besvaret 100\% fyldestgørende. Det skal dog siges, at det til en hvis grad var forventet, og at brugergrænsefladen ikke vil være optimal efter kun én test. Det anses som meget usandsynligt, at en så stor opgave, som at skabe en intuitiv brugergrænseflade, ville lykkedes perfekt i første forsøg på trods af den tilgængelige teori. \\

Gruppen må derfor affinde sig med, at betragte problemformuleringen som delvist besvaret indtil videre med det forbehold at flere usabilitytests, og derpå følgende forbedringsforløb, kunne lede problemformuleringen besvaret til fuldstændighed.
