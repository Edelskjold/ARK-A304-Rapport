% Jimmi. 19-05-2014 15:48
% Rune 20-05-2014 11:00

\subsection{Kvalitativ analyse}
\label{sec:kvalitativ_analyse}
Den kvalitative analyseform bygger på, at der indsamles dokumentation. Dette sker ofte igennem et dybdeinterview, som kan være med til at afdække relevante problemstillinger. Disse problemstillinger bliver belyst gennem en dialog mellem intervieweren og respondanten. Den kvalitative analyse går i sin enkelthed ud på at grave dybere ned i problemets kontekst. En af interviewets benævnelsesværdige fordele er, at spørgsmålene kan justeres løbende, hvorved intervieweren kan rette spørgsmålene i den retning, som virker påfaldende i den givne situation \cite{analyse_danmark}. \\

Et interview som analysemetode kunne være anvendeligt til at forstå en forenings problemstillinger. Interviewerens forståelse for klubbens problemer er på forhånd begrænsede og ved at føre en dialog, kan problemerne belyses. Herved kan intervieweren løbende opnå en øget forståelse for problemets omfang og faktorer. \\

Gruppen ønskede at arbejde med aktuelle samfundsrelaterede problemstillinger. Derfor valgte gruppen i starten at udsende en række spørgsmål til omkringliggende ro- og sejlklubber. Disse spørgsmål skulle kort afgøre, om hvorvidt klubben havde den ønskede størrelse, undervisningsforhold samt udlejningsforhold af materiel. (Se bilag \ref{bil:sporgeskema}). Aalborg Roklub besvarede spørgsmålene fyldestgørende, og ønskede ligeledes at medvirke i et interview med gruppen. Interviewet blev udført i bestræbelserne på at belyse klubbens generelle, organisatoriske eller ledelsesmæssige mangler eller problematikker, som muligvis kunne danne rammerne for projekts videre problembearbejdning. Dette vil næste afsnit komme nærmere ind på. 

\subsection{Foreningen, Aalborg Roklub}
\label{sec:foreningen_ark}
Aalborg Roklub blev stiftet i 1886, og er dermed en af Aalborgs ældste idrætsforeninger, og har 220 aktive medlemmere.\\

{\bf Interview af Aalborg Roklub}\\
Gruppen forberedte indledningsvis en række induktive spørgsmål, som havde til formål at øge forståelsen for klubbens problemstillinger. Hertil blev spørgsmålene opdelt i forskellige kategorier; såsom organisationen, uddannelsen, teknologi og økonomi. De kvalitative spørgsmål blev forberedt i henhold til den kvalitative analyse (afsnit \ref{sec:kvalitativ_analyse}) og Steinar Kvales model \cite{BOOK_KVALE}

\begin{itemize}
    \item \textbf{Tematisering}: Her opstillede gruppen til hvert af ovenstående kategorier et tilhørende formål. Dette blev gjort, fordi spørgsmålene skulle vælges med omhu, og skulle fremføre spørgsmålenes overordnede formål.
    \item \textbf{Design}: Gruppen strukturede spørgsmålene i en rækkefølge, hvor de mest relevante spørgsmål kom først. Dette var af hensyn til, at gruppen ikke vidste på forhånd, hvor lang tid Aalborg Roklub havde afsat til interviewet. Ydermere blev spørgsmålene opstillet i punktform for at fastholde en overskuelig struktur.
    \item \textbf{Interview}: Gruppen valgte en semi-struktureret interviewguide, hvilket betød, at gruppen havde forberedt en række spørgsmål, som skulle indlede til en forbedret forståelse for klubbens aktuelle problemstillinger.
    \item \textbf{Transskribering og Analyse}: Her foretog interviewerne en opsamling med resten af gruppen, hvor samtalen og notaterne blev opsummeret. Denne fase var elementær for at afgøre hvilke problemstillinger, som gruppen syntes var aktuelle i forhold til den videre udarbejdelse af projektet. 
    \item \textbf{Verificering}: Gruppen gennemarbejdede interviewets validitetsgrad for at afdække, om hvorvidt interviewet fik besvaret de spørgsmål, som der blev opstillet i tematiseringsfasen.
    \item \textbf{Rapportering}: Gruppen foretog sig en evaluering om hvilke aspekter af interviewet, der kunne være interessante at fremføre for læseren. Hertil fandt gruppen det meget interessant at arbejde med protokolsystemet, da gruppen så størst potentiale for udviklingen af et program, som kunne løse de fleste af Aalborg Roklubs problemstillinger.
\end{itemize}

Ovenstående er en skitsering af interviewets komposition. Efter interviewet analyserede gruppen på det indsamlede data, som kan ses i sin fulde længde i bilag \ref{bil:interview}. \\

Det analyserede materiale bliver behandlet og fremført i de kommende afsnit for derved at kunne fremføre og systematisere klubbens problemer.