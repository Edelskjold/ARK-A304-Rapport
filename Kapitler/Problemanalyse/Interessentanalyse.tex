%Skrevet af Jonas 2014-24-02 kl. 20:00
%Rettet af Rune 2014-02-03 kl. 16:00
\label{sec:interessentanalyse}
I forbindelse med interviewet og undersøgelsen af Aalborg Roklub, er der fundet forskellige grupper af interessenter. De efterfølgende afsnit vil forsøge at beskrive og afdække de forskellige interessentgruppers interesse i den generelle problemstilling. Heriblandt hvilken betydning en løsning af denne problemstilling vil have og hvilken effekt de forskellige moduler vil have på de enkelte interessentgrupper. På figur \ref{fig:interessentfigur} ses få udvalgte problemstillinger og interessenternes interesse i disse, for at give et nemt grafisk overblik over hvor stor en betydning de hver især har.

\figur{Figurer/interessentfigur.png}{Denne figur opstiller få, udvalgte problemstillinger og interessenternes interesse i disse. Dette er ikke alle behandlede problemstillinger. \figuregroup}{interessentfigur}{0.5}

\subsection*{Medlemmer}

%Rettet af Rune 27-02-2014 11:15
At være medlem af Aalborg Roklub indebærer, at man betaler kontingent, og at man benytter sig af det materiel, der er stillet til rådighed. Herigennem er det nødvendig at man benytter det eksisterende system i foreningens klubhus. Gennem interviewet (referat i bilag \ref{bil:interview}) med Aalborg Roklub er det gjort klart, at det eksisterende system allerede har sparet medlemmerne for meget tid, men at systemet også har potentiale for forbedringer mht. brugerfladen og de informationer, der stilles til rådighed for medlemmerne.\\

%Bookingsystem  -   Tidsbrug
Både det eksisterende program og projektets program skal benyttes af medlemmer til udlån af materiel. Foreningen, Aalborg Roklub, ønsker, at det kun skal være muligt at leje materiel ved fremmøde i klubrummet på den gældende dag. Dette er for at forhindre reservering af de bedste både, hvilket ville indskrænke brugen af disse til kun nogle enkelte medlemmer. Der falder også andre uheldige aspekter ind for denne type udlån, fordi skader kan opstå på bådene, og disse kan forhindre et medlem i at sejle på den forventede dag.\\

%Reparationsmodul   -   Tidsbrug
Det fremgår at den information, der er til rådighed for medlemmerne, ikke er tilstrækkelig, når materiel udlånes. I det eksisterende program er det ikke muligt at gennemskue, om en ønsket båd har mangler eller brug for reparation. Ligeledes er det fordelagtigt hvis medlemmerne selv kan registrere uheld og skader på båden ved hjemkomst. Disse mangler til det eksisterende program er tidskrævende og potentielle irritationsmomenter for medlemmerne. Der er derfor interesse i at løse dem.\\

%SMS-modul  -   Sikkerhed
Til trods for at der aldrig er sket større uheld med menneskelige skader i hele Aalborg Roklubs eksistens, er medlemmerne stadig åbne over for større tryghed ved roning. Foreningen har ikke nogle strikse regler angående hjemkomst - dog er det et krav, at man er tilbage inden mørkets frembrud. Til interviewet blev der vist interesse for et SMS-system, som ved mørkets frembrud kan underette relevante personer.

\subsection*{Bestyrelsen}

%Rettet af Rune 2014-01-03 kl. 16:00
Bestyrelsen er den gruppe af mennesker, der har det overordnede ansvar i foreningen. I Aalborg Roklub er det bestyrelsens ansvar at vedligeholde bådene, at opstarte ugentlige miniprojekt-grupper, at administrere medlemssystemet og at godkende langture. I forhold til at bestyrelsesarbejdet er ulønnet, så er det de færreste, der ønsker at bruge lang tid dagligt på at håndtere store administrative byrder. Derfor vil det være i bestyrelsens interesse at udvikle et system, som kan hjælpe dem.\\

%Reparationsmodul   -   Tidsbrug
I Aalborg Roklub registreres eventuelle skader eller mangler på det udlånte materiel manuelt. Bestyrelsen er påkrævet at undersøge det gældende materiel fysisk: De skal ud til båden og undersøge, om der hænger en reparationsseddel, og om skaden eller manglen gør båden ufunktionsdygtig. Indkoopereringen af et reparationsmodul ville hjælpe bestyrelsen med at få et overblik over skader og mangler fra ét sted. Ydermere kan modulet udvikles til at tilføje en specifik méngrad, så det kan vides, om en reparation er nødvendig før brug.\\

%Hjemmeside -   Tidsbrug
Gennem interviewet med Aalborg Roklub fremgik det at muligheden for at tilgå systemet hjemmefra, gennem enten en browser eller et program, var attraktivt. Meget af administration kræver ikke nødvendigvis fysisk tilstedeværelse i klubhuset, men det kræver dog tilgang til den data, som ligger i klubhusets database.\\

%Regnskabsmodul -   Omkostninger
Aalborg Roklubs udgifter og indtægter er ikke dækket af det nuværende program. Ajourførelsen med regnskabet sker gennem de opgørelser, som sendes via den bank, der er tilknyttet Aalborg Roklub. De årlige opgørelser koster som udgangspunkt 10.000 kr., og efterfølgende opkrævninger og opgørelser hertil har yderligere omkostninger for klubben. Et andet problem hertil er, at det er muligt for et ikke-betalende medlem at låne materiel. Det ville være tidskrævende og en udstråling på mangel af tillid, hvis hvert medlem skulle checkes manuelt for betalt kontingent. Hvorvidt systemet skal forhindre udlån af materiel er ikke fastlagt.

%Skrevet af Rune 2014-27-02 kl. 11:30
\subsection*{Samfundet}
Der er enkelte aspekter af programmets funktioner som både Aalborg Roklub og samfundet kan drage nytte af. Søværnets Operative Kommando, der er ansvarlig for redning af skibbrudene eller andre i nød på vandet, kan benytte systemets informationer i tilfælde af nødsituationer. Derudover, ved behov for beviser til en retslig proces, kan programmets informationer potentielt benyttes af myndighederne til bevismateriale. Interessenterne ligger uden for projektets direkte fokus, men er stadig nævnt, fordi de kan egne sig til en videreudvikling af programmet.\\

Samfundet er ekskluderet fra rapporten, da de tilhørende problemstillinger er for brede og abstrakte til, at projektetforløbets forventede ressourcer ikke kan bære det, samt projektets indsnævring til Aalborg Roklub ikke får udbytte ved en inddragelse af samfundet. 

%Skrevet af Rune 11-03-14 kl. 22:45
\subsubsection*{Opsummering}

\begin{itemize_small}
    \item Den primære målgruppe er bestyrelsen og hjælpere
    \item Der skal stadig tages hensyn til medlemmer, da disse også påvirkes
    \item Samfund ekskluderes fra interessenter
\end{itemize_small}

%På figur \ref{fig:interessentfigur} er en sammendragelse af nogle af de moduler, der er bearbejdet gennem interessentanalysen. Selvom, at alle tænkelige eller nævnte moduler gennem teknologianalysen ikke er inddraget, så viser figuren stadig, hvad der er projektets primære interessegruppe. Det er som udgangspunkt bestyrelsen og de frivillige hjælpere, der kan drage mest nytte af dette projekt - dog skal der stadig tages hensyn til medlemmer, da disse også vil blive påvirket.

