%Skrevet af Jimmi d. 03-03-14 kl. 15:00
%Jimmi d. 09-05-14 kl. 16:00
% Rune 20-05-2014 12:30
\subsection{Opsamling og analyse af indsamlet data} 
Igennem interviewet opnåede gruppen en god forståelse for, hvordan Aalborg Roklub er struktureret som en forening. Det var vigtigt for gruppen at få en intern forståelse for hvordan arbejdsopgaverne var fordelt på de forskellige bestyrelsesposter og, endvidere ligeså vigtigt at forstå hvordan Aalborg Roklubs medlemmer interagerede med deres nuværende IT-system. Det var ligeledes vigtigt, at gruppen indledningsvist opnåede den nødvendige forståelse for hvem og hvor mange, der anvendte Aalborg Roklubs IT-system for derved at forstå problemets aktualitet. Det kommende afsnit vil kort beskrive, hvad Aalborg Roklub varetager, deres størrelse, og hvordan deres bestyrelse er struktureret. \\

\subsubsection{Aalborg Roklub}
\label{sec:aalborg_robklub_organisation}
Aalborg Roklub er en sportsforening, som kun optager mænd. De har ingen børne/ungdomsafdeling, hvilket betyder, at de ikke optager medlemmer, som er under 18 år. Aalborg Roklub har over 200 aktive medlemmer, hvoraf majoriteten er over 50 år. Medlemmerne bruger aktivt Aalborg Roklub grundet deres brede materialebeholdning bestående af over 50 både, det sociale sammenværd og fælleslokaler. Medlemmerne betaler et årligt kontingent, som ubeskåret går til vedligeholdelsen af klubbens materiale og matrikel. Aalborg Roklub bliver til dagligt drevet af 7 frivillige bestyrelsesmedlemmer, som hver besidder en række arbejdsopgaver (se figur \ref{fig:aalborg1}).
\figur{Figurer/organisation/aalborg_roklub.png}{Foreningens, Aalborg Roklub, struktur \figuregroup}{aalborg1}{1.0}
Til vedligeholdelsen af Aalborg Roklubs materialer, dannes der jævnligt miniprojektgrupper, bestående af frivillige medlemmer. Dette giver de frivillige hjælpere frihed under eget ansvar, som Aalborg Roklub mener er den rette tilgang til tingene – nu hvor arbejdsopgaverne foretages frivilligt. Dette bekræfter ligeledes Aalborg Roklubs solide fællesskab, hvor alle aktivt ønsker at bidrage. 

%Skrevet af Mads d. 24/2-14
%Rettet af Jimmi d. 24-02-2014 kl. 22:54
%(Medlemmer) Redigeret af Martin. 03-03-2014 00:38
%\rfix{Kom ind på medlemmernes generelle beskrivelse før kontingent.}
%\subsubsection{Medlemmer}
%Alle medlemmer i Aalborg Roklub betaler årligt et kontingent. Dette kontingent bruger Aalborg Roklub på køb af nye materialer, projekter samt almen drift af klubben. \\
%Medlemmerne har egen nøgle til roklubben og har fri adgang til at komme døgnets 24 timer. Klubben har ligeledes et fitnesscenter, som de deler med to andre nærliggende klubber. Fitnesscentret er ligeledes også tilgængeligt døgnet rundt, og bliver flittigt brugt af medlemmerne såvel som bestyrelsen. \\

%De medlemmer, som ønsker at bidrage, kan komme i såkaldte miniprojektgrupper. Disse projektgrupper benytter Aalborg Roklub konstruktivt til alt fra udendørs vedligeholdelse på klubbens matrikel til reparation af beskadigede bådene. Det er hertil i klubbens interesse at efterkomme medlemmernes ønske om tidshorisont, som ofte korresponderer med arbejdsopgavens omfang og størrelse. Dette giver de frivillige hjælpere frihed under eget ansvar, som Aalborg Roklub mener er den rette tilgang til tingene – nu hvor arbejdsopgaverne foretages frivilligt. 

%\subsubsection{Bestyrelse}
%Aalborg Roklubs bestyrelse består af i alt syv personer, som tilsammen står for den dagligt drift. Da bestyrelsens repræsentative medlemmer frivilligt bruger tid på klubbens foretagende, er det også hér vigtigt, at arbejdsopgaverne er overkommelige. Den organisatoriske struktur kan ses på figur \ref{fig:aalborg1}

%\figur{Figurer/organisation/aalborg_roklub.png}{Foreningens, Aalborg Roklub, struktur \figuregroup}{aalborg1}{1.0}

%\noindent{\textbf{Formand}}\\
%Formanden fungerer som klubbens overhoved. Hans primære arbejdsopgaver består i, at bevare klubbens samlede overblik - uanset om dette skulle være klubbens økonomi, eller etableringer af miniprojektgrupper. Ligeledes repræsenterer formanden klubben regionalt og internationalt. Formanden har endvidere til opgave, at sørge for Aalborg Roklub overholder regler i forhold til dansk lovgivning - både på land og til vands. \\

%{\bf{Næstformanden}}\\
%Næstformanden er lige under formanden og er dermed stedfortræder for formanden. Ligeledes er næstformanden formandens sparingspartner. Næstformanden har til opgave at bestyre og planlægge vinteraktiviteter, samt at skabe et socialt sammenhold i Aalborg Roklub ved hjælp af aktiviteter. Sidst, men ikke mindst, er det næstformandens opgave at undersøge muligheder for at øge indtægterne i Aalborg Roklub, samt at se hvor der kan spares. \\

%{\bf{Økonomichefen}}\\
%Økonomichefen står i sammenråd med formanden og kommunikationschefen for den økonomiske afdeling af Aalborg Roklub. Økonomichefen har til ansvar at ajourføre Aalborg Roklubs regnskab og budgetter for kommende aktiviteter. \\

%{\bf{Kommunikationschefen}}\\
%Kommunikationschefen har ansvaret for at kommunikere via materialer (blade, foldere, opslag mm.) til Aalborg Roklubs medlemmer, såvel som at promovere roklubben til regionale medlemmer. Kommunikationschefen har ligeledes den praktiske opgave, at han skal fordele post til den retmæssige modtager i bestyrelsen. \\
%Ligeledes skal kommunikationschefen opdatere hjemmesiden (webredaktør) og udsende nyheder til mail-listen over medlemmerne. \\

%{\bf{Human Resource chefen}}\\
%HR-chefen har til ansvar at sørge for at Aalborg Roklubs bestyrelse besidder de påkrævede kompetencer. Dette er for at kunne udøve den optimale rosportslige kundskab overfor medlemmerne. Ydermere står HR-chefen for at indberette medlemskartoteker til DFfR og DGI\footnote{Dansk Forening for Rosport (DFfR) og Danske Gymnastik- \& Idrætsforeninger (DGI)}, samt foretage tilmelding til kurser for medlemmer og bestyrelsesmedlemmer. \\

%{\bf{Rochefen}}\\
%Rochefen har ansvaret for aktiviteter på vandet, herunder kaproning, langtursroning, ungdomsroning og kajakroning. Desuden har rochefen ansvaret for at aktiviteter bliver gennemført på en ordentlig og social måde. \\

%{\bf{Logistikchefen}}\\
%Logistikchefen har ansvaret for Aalborg Roklubs aktiver og materiel. Han sørger desuden for at materielet er i god stand, og at det bliver vedligeholdt. \\
%Ydermere har logistikchefen ansvaret for indkøb til køkkenet i klubhuset og salg derfra. \\