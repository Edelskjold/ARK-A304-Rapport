%\subsection{MVVM: Model-View-VievModel}
%Model-View-ViewModel, herefter udelukkende referet til som MVVM, er et designmønster, som oprindeligt blev præsenteret og udviklet, og er under stadig udvikling, af Microsoft, i forbindelse med udviklingen af WPF, og som er løst baseret på MVC\footnote{Model-View-Controller} og PM\footnote{Presentation model}\cite{mvvm_msdn}. MVVM er et designmønster med fokus på en fuldstænding opdeling af design og bagvedliggende logik og dette opnåes vha. data binding. Valget af MVVM, som det designmønster der vil blive benyttet, blev foretaget på baggrund af, at det er en praktisk måde at opdele en grafisk brugergrænseflade og funktionalitet, og dets sideløbende udvikling med WPF gør det nemt at implementere designmønstret i WPF. Efterfølgende vil være en kort beskrivelse af MVVM, da tankerne og ideerne bag er for mangfoldige til at inkludere i denne rapport.

%\subsubsection{Model}
%Modellen i MVVM beskriver den bagvedliggende datastruktur og adgangen til denne. Dette gælder både for klasser og forskellige collections og for adgangen til forskellige former for lager som fx databaser. Modellen beskriver hvilken rolle de forskellige data har i forhold til programmet, og den sørger ligeledes for simpel adgang til at læse og manipulere disse data, sådan at andre dele af programmet er i stand til at benytte sig af den funktionalitet uden at implementation skal testes overalt.

%\subsubsection{View}
%View i MVVM er den del, der beskriver alt hvad der relaterer sig til programmets GUI\footnote{Graphical User Interface} dvs. knapper, lister og andre GUI elementer. I forhold til MVVM bliver hvert enkelt "side" i et program betragtet som et separat View, og det spiller en rolle i forhold til opdeling i filer og klassehierarki, dette vil blive forklaret yderligere under ViewModel. Det der er anderledes ved MVVM i forhold til andre desigmønstre, er at et View ikke bør indeholde nogen logik eller data overhovedet. Et View skal udelukkende beskrive hvordan funktionalitet skal vises, hvorefter det via data binding binder sig til data og funktionalitet i en ViewModel, hvilket ofte medfører at der er et 1:1 forhold mellem Views og ViewModels.

%\subsubsection{ViewModel}
%En ViewModel er en abstraktion af et View. Som beskrevet under View så indeholder et View ikke nogen funktionalitet, men overlader dette til dens korresponderende ViewModel, så indeholder en ViewModel på samme måde ikke nogen beskrivelse hvordan funktionalitet skal vises, den indeholder udelukkende det data som skal udfylde dets korresponderende View, og den indeholder det logik som gør GUI'et interaktivt. Det betyder at den primære opgave for en ViewModel, er at være et bindeled mellem programmets View og Model. En ViewModel skal sørge for at bringe data på en form, som kan bindes til og præsenteres i et View, og den skal sørge for at videregive kommandoer til en Model, når brugeren trykker på en knap.

%\subsubsection{Opsummering}
%Denne opdeling af programmet som MVVM beskriver, betyder at GUI'en ingen logik indeholder, og dermed ikke indeholder noget der skal testes udover GUI'en selv. Ligeledes er et View ikke klar over at Modellen eksisterer, hvilket gør det muligt at ændre i funktionalitet uden at begynde at ændre i Views.\\
%I MVVM mønstet er programmets ViewModel og Model ligeledes ikke klar over at View'et eksisterer, og Modellen er ikke klar over at ViewModellen eksisterer. Som nævnt så er det nemmere at teste små bidder af programmet, hvilket medfører at denne løse kobling af programmets forskellige dele gør det nemmere at teste.