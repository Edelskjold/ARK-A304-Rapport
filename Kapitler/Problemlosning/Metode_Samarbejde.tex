% Rettet af Rune, 10-05-2014 12:10
% Rettet af Martin, 19-05-2014
\subsection{Inddragelse af brugere}
\label{sec:metode_samarbejde}
Udviklingen af det ønskede program er afhængig af "kunden"\mbox{}: Aalborg Roklub. Programmets funktionsdygtighed skal derfor afspejles med de krav og forventninger, der stilles af roklubben. Hertil ses det som mest hensigtsmæssigt at benytte sig af en kombination af metoder.\\

% Mikkel, 29-04-2014
Projektets udvikling tager base i Vandfaldsmetoden, som herefter er tilpasset for at opnå en form for agil\footnote{At kunne iterere henover noget} arbejdsmetode. Det har været nødvendigt at modificere metoden, da vandfaldsmetoden ofte har den ulempe at man ikke vender tilbage til noget efter det er testet - det er derfor ikke nemt at rette eventuelle fejl eller forslag til designændringer. \cite{Vandfaldsmetoden}

\figur{Figurer/arbejdsproces.jpg}{Tilpasset arbejdsproces \figuregroup}{arbejdsproces}{1.0}

Figur \ref{fig:arbejdsproces} viser hvordan denne arbejdsproces er opbygget. Indledningsvist bliver der foretaget et interview, for at afdække de krav og ønsker kunden har til produktet. Disse information kan bruges til at analysere og planlægge hvordan og hvornår disse ting skal gøres. Derefter vil der være en eller flere iterationer, hvor design og kodning fremvises. Inden en sådan iteration vil udviklerne (gruppen) have nogle antagelser, som igennem feedback med kunden afklares. Nye ønsker vil også komme på banen her, hvor der igen skal analyseres og planlægges. Dette gentages indtil det ønskede resultat er opnået. Når disse skridt er overstået, vil der først blive udført tests af udviklerne, for at sikre sig at disse er tilfredse, samt fange åbenlyse fejl. Derefter skal produktet igennem en usability-test hos roklubben, hvor systemet skal afprøves i praksis. Dette kommer der noget feedback ud af, som efter en analyse og planlægning går tilbage til skridtet for Design \& Kodning. Afslutningsvist er det færdige produkt klar til at blive udgivet. Gruppen tilstræber sig at overholde denne arbejdsproces så godt som muligt i forhold til tidsrammerne. \\

Til hver iteration er følgende gældende:
\begin{itemize_small}
    \item Det er Aalborg Roklub, der skal anvende programmet
    \item Det er Aaalborg Roklubs tilfredshed ved anvendelsen af programmet, der                endeligt afgører, om programmet er brugervenligt
    \item Det er bestyrelsen i Aalborg Roklub, der kan give udtryk for, om programmet           opfylder de ønskede funktioner
\end{itemize_small}

% Rettet af Rune, 10-08-2014 12:30
%Selvom programmet læner sig meget op af hvad Aalborg Roklub mener, er det vigtigt at bemærke: Det er først og fremmest gruppens krav til projektet, der skal tilfredsstilles.