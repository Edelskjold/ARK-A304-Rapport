%Jonas, 02-04-2014
\subsection{Modulær opbygning}
For at sikre at programmet let kan udbygges, vil det blive opbygget af individuelle moduler, som kan tilføjes eller fjernes efter ønske. Hvis dette gøres ordentligt, vil det sikre at eventuelle tilføjelser ikke vil kræve store ændringer, da nye moduler blot kan udarbejdes og tilføjes. I praksis betyder dette, at programmet skal bygges op omkring et framework, som dynamisk tilbyder muligheder for at udvide brugergrænsefladen ved at  tilføje funktionalitet - uden at dette påvirker allerede eksisterende funktionalitet.
Denne mulighed for dynamisk at tilføje funktionalitet giver dog også en problemstilling i forhold til sikkerhed, da uhindret adgang til fx filsystemet potentielt vil gøre det muligt for udvidelser at tilgå private oplysninger, som måtte ligge på brugerens computer. Dette stiller et krav til at udvidelsernes adgang til forskellig funktionalitet begrænses i et sådant omfang, at det stadig er muligt for udvidelserne at fungere, men at udvidelserne ikke har adgang til kritiske dele af systemet som fx filsystemet og netværket.\\

Udover de fordele og ulemper en modulær opbygning har i forhold til funktionalitet, så giver det også en klar fordel i forhold til at udvikle et fejlfrit program. En modulær opbygning kræver mere arbejde fra starten af, men det er en investering, der giver afkast, når der tilføjes funktionalitet, da en modulær opbygning også gør det muligt at teste individuelle dele af programmet separat fra resten. Det er betydeligt nemmere at teste individuel funktionalitet direkte, i stedet for at gøre dette via anden funktionalitet, som fx brugergrænsefladen. Tests via brugergrænsefladen er svære at automatisere og skal derfor udføres manuelt, hvilket er uhensigtsmæssigt og meget tidskrævende. Dette ønske om en løs kobling mellem brugergrænsefladen og den bagvedliggende funktionalitet fører derfor videre til næste punkt.
