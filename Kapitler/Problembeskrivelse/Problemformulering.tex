%Rettet af Rune d. 03-03-14 kl. 15:30
%Rettet af Jimmi d. 11-03-14 kl. 23:12
%Fandme en lækker formulering!
\section{Problemformulering}
Gruppen har valgt at opsætte en række fokuspunkter, som skal danne rammerne for den opstillede problemformulering. Disse fokuspunkter er valgt på baggrund af tidligere opstillede problemstillinger, som blev fremført i problemanalysen (afsnit \ref{sec:problemanalyse_indledning}), og ydermere ud fra den beslutningstagende som der blev nævnt tidligere i problemafgrænsningen (afsnit \ref{sec:problemafgraensning}). \\

Gruppen har valgt at fokusere på administrativt overblik, som skulle ledsage til tidsbesparelser i de administrative arbejdsopgaver. Ydermere vil der være fokus på brugergrænseflade, og hvorledes dette konstrueres hensigtsmæssig, så funktioner forekomme intuitivt og letsindigt for samtlige af klubbens medlemmer – uafhængig af alder og IT-kyndighed. Tilgængeligheden til systemet skal optimeres, hvortil det er vigtigt at konstruere en løsning, som uafhængig af lokation, kan tilgås. Hertil er det tredje og sidste fokuspunkt, tilgængelighed. \\

Med det udgangspunkt, ser gruppens problemformulering således ud: \\

\textbf{Hvordan kan der på baggrund af den opstillede problemanalyse konstruereres en hensigtsmæssig softwareløsning, som foruden at hjælpe Aalborg Roklubs bestyrelse er med til at bevare overblikket over udlånt materiel, også vil ledsage til forøget brugerfunktionalitet blandt klubbens medlemmer?}

%Arbejdet i oklubberne består af frivilligt arbejde, som begrænser den enkelte persons tidsramme for opgaverne. Kan der konstrueres en softwareløsning, så dette tidsforbrug mindskes mest muligt i forhold til de administrative opgaver, der skal udføres af de ansvarlige i en ro- og sejlklub?