%WIP, Rune startet 11:30
%Mikkel, 03-03-2014 21:55
%Rettet af Jimmi d. 11-03-14 kl. 23:12
\section{Problemafgrænsning}
\label{sec:problemafgraensning}

Dataindsamlinge (afsnit \ref{sec:dataindsamling}) belyste specielt Aalborg Roklub og deres nuværende IT-systemer. Her kunne der erfares at Aalborg Roklub har et velfungerende system, som anvendes flittigt af klubbens medlemmer – deres såkaldt protokolsystem, som kan ses i afsnit \ref{subsec:itSystem}. Ydermere havde Aalborg Roklub et administrativt IT-system - medlemskartoteket, som kan ses i afsnit \ref{subsec:itSystem}, der udelukkende blev anvendt af bestyrelsesmedlemmerne. Medlemskartoteket blev anvendt til at lagre forskellige oplysninger, såsom medlemmernes oplysninger, kontingent-opgørelser og uddannelsesniveauer. \\

På trods af at Aalborg Roklub's IT-systemer fungerede tilstrækkeligt, indeholdte programmet stadig en række mangler og problemstillinger som kan ses i afsnit \ref{sec:foreningen_ark}. Her kunne især blanket-indregistrering nævnes, hvor Aalborg Roklub på nuværende tidspunkt håndterede alle langturs-anmodninger, skadeindmeldelser og anmærkninger af beskadigede både på tidskrævende manuelle fremgangsmåder. Disse fremgangsmåder var en arbejdsmæssige belastning for de pågældende bestyrelsesmedlemmer, som varetager klubbens interesser uden nogen former for aflønning. Dette sætter yderligere krav til IT-systemernes databasestruktur, som gerne skulle ledsage til øget overblik hos bestyrelsen; hvorved klubbens interne materiel, medlemmer og som tidligere nævnt, blanketter, gerne skulle arkiveres sikkert og lettilgængeligt. Aalborg Roklub ønskede i særlig grad, at trygheden for medlemmer blev forøget (afsnit \ref{subsec:itSystem_opsummering}). Hér viste det sig at langturs- og skadeblanketterne ikke altid blev udfyldt af medlemmerne, hvilket kunne have katastrofale konsekvenser ved kæntringer eller andre uheld. Men ved at gøre det nemmere og mere intuitivt for medlemmerne at udfylde disse blanketter, kunne det sandsynligvis føre til, at flere ville udfylde blanketterne. Når der skal foretages administrative arbejdsopgaver, er bestyrelsesmedlemmerne på nuværende tidspunkt tvunget til at være fysisk tilstede i Roklubbens lokaler, hvorved de havde et ønske om at kunne foretage nogle af disse opgaver hjemmefra. Dette stiller krav til databasestrukturen (afsnit \ref{sec:Databaseteknologi}), da en digitalisering af nogle af opgaverne - såsom godkendelse af blanketter - vil afhænge af, at databasen kan tilgås online. \\

Gruppen har valgt, i samarbejde med Aalborg Roklub, at fokusere på problemstillingerne forbundet med protokolsystemet, og administration af dette. Dette valg blev truffet i samvirke med Aalborg Roklub, som havde nogle konkrete forbedringsforslag til brugergrænsefladen og programmeringsstrukturen. Ydermere var det også her gruppen kunne udarbejde elementære funktioner, som at begrænse medlemmernes valgmuligheder – baseret på deres uddannelsesniveauer - og implementere funktioner, som at digitalisere blanketterne. \\
Analysens tekniske afsnit (\ref{sec:teknologianalyse}) fokuserede hovedsagligt på de eksisterende teknologier, for at få et indblik om hvorvidt disse var i stand til at løse de opstillede problemstillinger, som er forbundet med protokolsystemet. Her kunne gruppen udlede, at de eksisterende systemer forsøger at henvende sig mangfoldigt, og på kryds og tværs af sportsgrene. Usability- og trunktesten (afsnit \ref{sec:Usability} og \ref{sec:trunktesting}) belyste dog, hvordan brugeren interagerer med systemer, som kan bruges konstruktivt under selve konstruktionen af brugergrænsefladen. 
Disse afgrænsninger leder op til den samlede problemformulering, som danner grundlag for problemløsningen.\\



% Martin: Jeg synes afgrænsningen skal skrives om, så vi nævner de forskellige problemstillinger vi er stødt på i løbet af problemanalysens afsnit, og hvordan vi har afgrænset i disse. Det er for uspecifikt lige nu.
% Kommentarer skal foregå vha. fixme's - ikke udkommenterede ting i "koden"

%Problemanalysen har gennemgået de problemstillinger, som indgår i administrationen med ro- og sejlklubber. Analysens tekniske afsnit er hovedsaligt fokuseret på de eksisterende brugerflader og moduler, der tilbydes af Klubmodul, ForeningLet og Aalborg Roklubs nuværende system. Generelt er det ForeningLets løsning, der opfylder de fleste behov, der kan stilles af en ro- eller sejlklub. Dog er der stadig ulemper\mafix{Hvilke ulemper. Vi skal være mere specifikke.}, der forhåbentligt kan revideres for at opnå en bedre løsning (afsnit \ref{sec:Usability}).\\

%Under analysens forløb har samarbejdet med Aalborg Roklub skabt en stor interesse for en specifik løsning, da projektets dataindsamling kan udføres i en direkte og iterativ form (afsnit \ref{sec:dataindsamling}). Med iterativ menes der, at produktet kan udvikles løbende med flere revisioner, hvorpå hver revision er en forbedring af den forrige. Dette skal ske vha. den kritik, som Aalborg Roklub kan stille til rådighed. Som det kan forventes af en iterativ behandling af produktet, vil produktet derfor løbende blive opdateret for at opnå det bedst mulige resultat for Aalborg Roklub, og dette er som udgangspunkt også målet.\\

%Lort
%I modulbeskrivelsesafsnittet (\ref{sec:modulbeskrivelse}), er det afklaret hvilke moduler, der er nødvendige eller nyttige for Aalborg Roklub. Her indgår bl.a. medlemsstyring, regnskabsføring og materieludlejning samt administrative opgaver, der i øjeblikket er dækket af deres nuværende program. Hertil er Aalborg Roklub selv kommet med forslag til forbedringer af programmet, der tilfælles berører den mængde af tid, som de frivillige hjælpere og bestyrelsen skal sætte af til at udføre den daglige og årlige administration. Efter en analyse af interessenter i afsnit \ref{sec:interessentanalyse} viste det sig også som den mest aktuelle gruppe at fokusere på.\\

%Disse afgrænsninger leder op til den samlede problemformulering, som danner grundlag for selve problemløsningen.

