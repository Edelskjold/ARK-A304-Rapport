\section*{Læsevejledning}
% Rune 18-05-2014 14:50
Følgende læsevejledning og det foregående forord er inspireret af P1-projektet "Automatisér dit hjem" \cite{p1_projekt}.\\

{\bf Kildehenvisning} \\
Ved kildehenvisninger anvendes Vancouver-metoden. Dette bliver brugt ud fra hvert afsnit eller citat, som henviser til en specifik kilde i litteraturlisten. Her vil der stå et tal indkapslet i hårde parenteser. \\

\textit{Eksempel: Påstand/citat\emph{[1]}}.\\

Hvis der i et afsnit er anvendt store mængder information fra en kilde, står kilden efter punktummet i afslutningen af afsnittet.\\

\textit{Eksempel: Afsnit 1}\\

Litteraturlisten er kompositorisk opsat således, at kilderne er placeret efter den numeriske orden, de er anvendt i teksten. Ud fra deres repræsentative nummerering er alle relevante informationer associeret. \\

\textit{Eksempel: Forfatter(e), Titel på artikel/afsnit, sider med relevant information, bogens titel, redaktør, forlag, udgivelsesårs og ISBN-nummer.} \\

Hvis nogle af informationerne mangler, som f.eks. forfatterens navn, udelades disse informationer i kildebeskrivelsen. \\

{\bf Figurhenvisning} \\
I rapporten vil der løbende blive refereret til figurer eller illustrationer. I den kontekstuelle sammenhæng, hvor figurerne anvendes, vil dette være angivet på følgende måde: \textit{Afsnit.nummer} \\

\textit{Eksempel: 2. figur i 3. afsnit vil være angivet med følgende referencenummer: 3.2}. \\

Under figuren vil figurenes referencenummer samt en dertilhørende figurbeskrivelse, som forklarer figurens relevans være påført den pågældende figur. \\ 

\textit{Eksempel: "Erklæring som en figur er kildemateriale til"}. Se figur \ref{fig:FigurEksempel}.\\
\figur{Figurer/Figureksempel.png}{Eksempel på figur beskrivelse.}{FigurEksempel}{0.3}

{\bf Fodnote henvisning}\\
Fodnoter benyttes til at inkludere yderligere informationer om et specifikt begreb, fagudtryk eller forkortelse. Fodnoterne beskriver sjældent vitale informationer for forståelsen af det omtalte. En fodnote ses ved et lille tal som henviser til sit modstykke i bunden af siden, hvor den vedhæftede tekst er vist. \\

\textit{Eksempel: Fodnote\footnote{Eksempel på fodnote}}.\\

Første gang et specielt fagudtryk eller begreb bliver benyttet, introduceres det enten i en fodnote eller i den kontekstuelle sammenhæng. \\

{\bf Meta-tekst}\\
Hvert afsnit indledes med en meta-tekst som er anført i kursiv. \\

\textit{Eksempel: Dette afsnit indeholder...}. \\

\textbf{Opsummeringer}\\
Slutningen af ethvert afsnit følges af en opsummering. Dette er opstillet i nogle få og præcise punkter, som beskriver, hvad der kan konkluderes ud fra afsnittet.\\

\textit{Eksempel:
\begin{itemize_small}
    \item Det er vigtigt at...
    \item Det er ikke muligt at...
\end{itemize_small}}

{\bf Tabeller}\\
Der vil løbende blive fremvist tabeller, som anvendts til at give læseren en bedre visualisering i kontekst med tekstens indhold.

\begin{table}[h]
    \centering
    \begin{tabular}{ l | l }
        \textbf{Overskrift} & \textbf{Overskrift} \\
        \hline \hline
        A & B \\
    \end{tabular}
    \caption{\textit{Eksempel på en tabel. \tabelgroup}}
    \label{tab:abc}
\end{table}

{\bf Kodeeksempler}\\
Der vil i problemløsningensafsnittet løbende blive refereret til kodeeksempler.\\
\textit{Eksempel:}

\CSharp{Kode/eksempel.cs}{Beskrivelsen af den viste kode er placeret her}{Code_example}