% Skrevet,  Mikkel  03-03-2014
% Rettet,   Rune    08-04-2014
% Rettet af Mikkel 24-04-2014
% Rune 18-05-2014 15:50
% Jimmi 19-05-2014 15:41

\section{Rapportens struktur}


\textbf{Problemanalyse}\\
Dette kapitel har til formål at belyse Aalborg Roklub’ eksisterende IT-systemer, hvorved fejl, mangler og problemstillinger kan forefindes. De tilgængelige IT-systemer på markedet bliver også undersøgt, for at kunne afgøre om hvorvidt disse kan løse Aalborg Roklubs mangler og fejl. Hertil findes følgende afsnit:
    \begin{itemize}
    \item  \textbf{Dataindsamling}: I udarbejdelsen af projektet blev Aalborg Roklub interviewet. Overvejelser, resultater og beskrivelse af klubben findes i dette afsnit

    \item \textbf{Teknologianalyse}: Har til formål at belyse kravene i teknologien, finde eksisterende teknologi og derefter udføre tests på disse med fokus på brugervenlighed

    \item \textbf{Interessentanalyse}: Anvendes til at analysere forskellige interessenter for derefter at finde ud af, hvilke målgrupper der kunne have interesse i produktet
    \end{itemize}

\textbf{Problembeskrivelse}\\
Kapitlet indeholder en opsamling af alle foregående afgrænsninger, som er blevet foretaget undervejs i problemanalysen. Disse afgrænsninger bliver afslutningsvis konkretiseret til en problemformulering. Problemformuleringen danner grundlaget for problemløsningen. \\

\textbf{Problemløsning}\\
I dette kapitel behandles den opstillede problemformulering. 

    \begin{itemize}
    \item \textbf{Kravspecifikationer}: Dette afsnit gennemgår en kort overgang fra problembeskrivelsens abstrakte problemformulering til et række funktionelle og non-funktionelle krav til produktløsningen
    \item  \textbf{Teori}: Afsnittet berører de teoretiske aspekter indenfor design og databaser, som bruges til at afgøre hvordan disse skal konstrueres mest hensigtsmæssigt
    \item \textbf{Metode}: Afsnittet redegør for de udviklingsmetoder og redskaber, som er anvendt under projektudarbejdelsen
    \item \textbf{Implementering}: Afsnittet viser dele af projektets endelige løsning, hvordan disse er implementeret, og om de har opfyldt de opstillede krav til produktet
    \end{itemize}


\textbf{Diskussion}\\
Afsnittet tager projektets afgrænsnings- og funktionalitetvalg op til diskussion, for dermed at redegøre for, om der blev foretaget det korrekte valg. \\

% Rune 20-05-2014 09:35
\textbf{Konklusion}\\
I konklusionen foretages der en opsamling af gruppens endelige produkt og en belysning på, om hvorvidt produktet løser de opstillede kravspecifikationer. \\

\textbf{Perspektivering}\\
Perspektiveringen opstiller interessante udvidelses-potentialer og om hvordan, at programmet kan blive tilpasset, så det kan bruges til andre klubber eller foreninger end blot Aalborg Roklub.
