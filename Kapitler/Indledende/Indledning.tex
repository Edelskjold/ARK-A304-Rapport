%Fjerde forsøg - Skrevet af Jonas d. 03-03-2014
%Rettet af Jimmi. d. 03-03-2014 kl. 12.34
%Rettet af Søren. d. 09-03-2014 kl. 15.12
%Rettet af Rune 11-03-14 kl. 12:45
%Rettet af Jimmi. d. 13-03-2014 kl. 21.24
% Rune 20-05-2014 09:30

\label{sec:indledning}
I Danmark er der registreret omkring 101.000 organisationer, hvoraf de 18.000 af dem er sportsforeninger af forskellig art. En forening er en samling af mennesker, der er organiseret efter en fælles interesse, og kan inddeles efter flere kriterier såsom struktur, formål og geografisk placering. \cite{Antal_Frivillige}\cite{DefinitionForening} Fælles for alle typer af foreninger er, at selve foreningen er en tidskrævende beskæftigelse - både i forhold til den aktivitet, som foreningen omfatter, og i forhold til det administrative arbejde, der er påkrævet for at drive en forening.\\

Umiddelbart ville det administrative arbejde ikke være et problem, hvis personerne der varetog disse opgaver, var aflønnede. Langt de fleste foreninger er dog drevet af frivillige mennesker, som har et fuldtidsarbejde ved siden af. Dette medvirker til, at arbejdsopgaverne ofte fordeles jævnbyrdigt på foreningens bestyrelsesposter for at kunne efterkomme de tidsmæssige begrænsninger.\\

Det administrative arbejde består bl.a. i, at foreningen årligt skal kunne dokumentere økonomiske transaktioner, medlemsoplysninger og andre bogføringsdokumenter, hvor mange foreninger også har en række resourcer til rådighed, som ligeledes skal administreres. Disse resourcer kan afhængigt af foreningens type bestå af forskellige områder og/eller materiel, som foreningen ejer eller bestyrer, og ønsker at stille til rådighed for deres medlemmer. Dette giver en del arbejde, når resourcerne forventes at blive ligeligt fordelt, og når medlemmerne skal stilles til ansvar for resourcerne, som de bruger.\\

Kigger man på en bestemt type forening, nærmere bestemt ro- og sejlklubber, har de nogle helt konkrete problemstillinger i forhold til udlån og administration af både. Klubben har materiel i form af både og lokaler, som kan benyttes af medlemmerne. Især i forbindelse med bådudlån, er der flere administrative opgaver, som skal varetages. Udlån skal registeres, så ansvaret for båden bliver placeret hos de rigtige, beskadigede både skal markeres og repareres, og ture skal registreres, så der er et overblik over hvem, der er på vandet. \\

Alle de førnævnte arbejdsopgaver er noget, man kunne forestille sig, at et IT-system kunne afhjælpe. Hvis foreningen stoler på dets medlemmer, er der ikke noget i vejen for, at medlemmerne selv kunne varetage de førnævnte opgaver i forhold til registrering og markering, hvis blot de fik hjælp af et IT-system, som på samme tid kunne give et overblik over materiel.\\

På baggrund af denne indledning, kan følgende initierende problemstilling dermed opstilles:\\

\textit{Hvordan kan et IT-system hjælpe ro- og seljklubber med at administrere materiel og medlemmer?}
