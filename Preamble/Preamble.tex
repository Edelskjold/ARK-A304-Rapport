% Typen af dokumentklassen
\documentclass[a4paper,11pt,twoside]{report}

% Sæt encoding så indstillingerne kan loades korrekt
\usepackage[utf8]{inputenc}

% Indlæs indstillinger
% Settings - udkommenter for at slå fra
%\newcommand*{\draft}{} % Skal der compiles som draft
\newcommand*{\CMDCode}{} % Kode kommandoer
\newcommand*{\CMDFixMe}{} % Aktiver fixme

% Sæt dokumentinformationerne
\newcommand{\rapportnavn}{Administration af bådudlejning}
\newcommand{\gruppen}{A304}
\newcommand{\gruppenummer}{A304}
\newcommand{\afldate}{Maj 2014}

% Ekstra kommandoer
\newcommand{\figuregroup}{(Figuren er fremstillet af gruppen)}
\newcommand{\tabelgroup}{(Tabellen er fremstillet af gruppen)}
\newcommand{\codegroup}{(Kodestykket er fremstillet af gruppen)}
\newcommand{\screenshotgroup}{(Skærmbillede er taget af gruppen)}
\newcommand{\lastseen}[1]{Sidst set d. #1}           

% Indlæs packages
\usepackage{tabularx}                   % Til tabeller
\usepackage{fourier}                    % Font
\usepackage[danish]{babel}              % Dansk sprog
\usepackage{a4}                         % A4 format
\usepackage[margin=3.0cm]{geometry}     % Til at sætte margin
\usepackage{fancyhdr}                   % Pæn header og footer
\usepackage{lastpage}                   % Til at finde sidste side
\usepackage{hyperref}                   % Referencer i PDF dokumentet og referencer til labels
\usepackage{float}                      % Til at sætte figurer korrekt
\usepackage{url}                        % Til at lave URL
\usepackage{titlesec}                   % Til at ændre på størrelsen på sections
\usepackage{xr}                         % Gør det muligt at cross referere i andre filer
\usepackage{amsmath}                    % Matematik-ting
\usepackage{mathtools}                  % Flere matematik-ting
\usepackage{multirow}                   % Multirows i tabel
\usepackage{threeparttable}             % Til fodnoter i tabeller
\usepackage{natbib}                     % Udvidelse af kilder
\usepackage[toc,page]{appendix}         % Til bilag
\usepackage{enumitem}                   % Til lister
\usepackage{graphicx}                   % Billeder
\usepackage{pdfpages}                   % PDF sider
\usepackage{color}                      % Farver
\usepackage{soulutf8}                   % Til highlight
\ifdefined\draft\usepackage{Preamble/Packages/currfile}\fi% Til indsættelse af filnavn - kun hvis draft er til

% Dokument informationskommandoer
\newcommand{\currentpage}{\thepage}     % nuværende side
\newcommand{\numpages}{\pageref{LastPage}}% antal sider
\newcommand{\sidetal}{Side {\currentpage} af {\numpages}} % side informationer
\newcommand{\pagetitle}{\rightmark}     % nuværende sektion

% Definerer titel osv.
\title{\rapportnavn}
\author{\gruppen}
\date{\afldate}

% Setup af header og footer
\fancypagestyle{plain}{
    % Nulstil header og footer
    \fancyhead{}
    \fancyfoot{}

    % Setup header
    % Left / Right - Even / Odd
    \fancyhead[LE]{\rapportnavn}        % Lige sider
    \fancyhead[RO]{\pagetitle}          % Ulige sider

    % Sæt footer op
    \fancyfoot[LE]{\sidetal}            % Lige sider
    \fancyfoot[RO]{\sidetal}            % Ulige sider
    
    % Sæt kun filnavn på en draft
    \ifdefined\draft
        \fancyfoot[LO]{\currfilepath}   % Ulige sider
        \fancyfoot[RE]{\currfilepath}   % Lige sider
    \fi
}

% Sætter pagestyle til plain - header og footer
\pagestyle{plain}

% Fjerner ligegyldig whitespace - men rykker ting op så vi ikke har teksten på slut
\raggedbottom

% Fjerner punktummet efter sektion-nummer i header (2.2. XXX --> 2.2 XXX)
\renewcommand\sectionmark[1]{%
  \markright{\MakeUppercase{\textbf{\thesection}\ #1}}}

% Fjerner afstand mellem kapitel og sætter størrelsen på skriften - se evt. her: http://tex.stackexchange.com/questions/63390/how-to-decrease-spacing-before-chapter-title
\titleformat{\chapter}[display]{\normalfont\huge\bfseries}{\chaptertitlename\ \thechapter}{20pt}{\Huge}
\titlespacing*{\chapter}{0pt}{0pt}{20pt}

% Indent til og fra
\newcommand{\enableIndent}{\setlength\parindent{6pt}}
\newcommand{\disableIndent}{\setlength\parindent{0pt}}
\disableIndent

% Opretter subsubsubsection og subsubsubsubsection
\newcommand{\subsubsubsection}[1]{\noindent\paragraph{#1}\mbox{}\\}
\newcommand{\subsubsubsubsection}[1]{\noindent\subparagraph{#1}\mbox{}\\}

% Skriftstørrelse på sections
\titleformat*{\section}{\LARGE\bfseries}
\titleformat*{\subsection}{\Large\bfseries}
\titleformat*{\subsubsection}{\large\bfseries}
\titleformat*{\paragraph}{\normalsize\bfseries}
\titleformat*{\subparagraph}{\normalsize\bfseries}

% Definerer antallet af overskrifter der har tal, samt antallet af overskrifter i indholdsfortegnelsen
\setcounter{secnumdepth}{2}             % Antallet af overskrifter der har et nr - dybden
\setcounter{tocdepth}{2}                % Antallet af overskrifter i indholdsfortegnelse - dybden - eks. 1.1.1

% Orddeling
\hyphenation{}

% Ændrer indstillingerne for caption for figurer, tabeller og kodestykker - tilføjer bla. kusiv og gør typenavnet bold (eks. Figur 1.1: bliver fed)
\usepackage[font=small,format=plain,labelfont=bf,up,textfont=it,up]{caption}

% FixMe
\ifdefined\CMDFixMe
    \input{Kapitler/Indledende/Titelblad/Index.tex}
\input{Kapitler/Indledende/Forord.tex}
\input{Kapitler/Indledende/Lasevejledning.tex}
\input{Kapitler/Indledende/Indholdsfortegnelse.tex}

\chapter{Indledning}
\input{Kapitler/Indledende/Indledning.tex}
\input{Kapitler/Indledende/Struktur.tex}


\clearpage
\section*{Forord}
% Rune 18-05-2014 14:40
% Rune 20-05-2014 09:20
Denne rapport er udarbejdet af en gruppe software-studerende på 2. semester på Det Teknisk-Naturvidenskabelig Fakultet på Aalborg Universitet.\\

I projektet har gruppen taget udgangspunkt i AAU-modellen, som tager afsæt i problemorienteret projekt- og gruppeorganiseret læring, hvilket grundlæggende består i en læringsproces, der indledes med en problemanalyse. Denne analyse danner herefter grundlag for rapportens videre bearbejdning af problemets løsning.
\section*{Læsevejledning}
% Rune 18-05-2014 14:50
Følgende læsevejledning og det foregående forord er inspireret af P1-projektet "Automatisér dit hjem" \cite{p1_projekt}.\\

{\bf Kildehenvisning} \\
Ved kildehenvisninger anvendes Vancouver-metoden. Dette bliver brugt ud fra hvert afsnit eller citat, som henviser til en specifik kilde i litteraturlisten. Her vil der stå et tal indkapslet i hårde parenteser. \\

\textit{Eksempel: Påstand/citat\emph{[1]}}.\\

Hvis der i et afsnit er anvendt store mængder information fra en kilde, står kilden efter punktummet i afslutningen af afsnittet.\\

\textit{Eksempel: Afsnit 1}\\

Litteraturlisten er kompositorisk opsat således, at kilderne er placeret efter den numeriske orden, de er anvendt i teksten. Ud fra deres repræsentative nummerering er alle relevante informationer associeret. \\

\textit{Eksempel: Forfatter(e), Titel på artikel/afsnit, sider med relevant information, bogens titel, redaktør, forlag, udgivelsesårs og ISBN-nummer.} \\

Hvis nogle af informationerne mangler, som f.eks. forfatterens navn, udelades disse informationer i kildebeskrivelsen. \\

{\bf Figurhenvisning} \\
I rapporten vil der løbende blive refereret til figurer eller illustrationer. I den kontekstuelle sammenhæng, hvor figurerne anvendes, vil dette være angivet på følgende måde: \textit{Afsnit.nummer} \\

\textit{Eksempel: 2. figur i 3. afsnit vil være angivet med følgende referencenummer: 3.2}. \\

Under figuren vil figurenes referencenummer samt en dertilhørende figurbeskrivelse, som forklarer figurens relevans være påført den pågældende figur. \\ 

\textit{Eksempel: "Erklæring som en figur er kildemateriale til"}. Se figur \ref{fig:FigurEksempel}.\\
\figur{Figurer/Figureksempel.png}{Eksempel på figur beskrivelse.}{FigurEksempel}{0.3}

{\bf Fodnote henvisning}\\
Fodnoter benyttes til at inkludere yderligere informationer om et specifikt begreb, fagudtryk eller forkortelse. Fodnoterne beskriver sjældent vitale informationer for forståelsen af det omtalte. En fodnote ses ved et lille tal som henviser til sit modstykke i bunden af siden, hvor den vedhæftede tekst er vist. \\

\textit{Eksempel: Fodnote\footnote{Eksempel på fodnote}}.\\

Første gang et specielt fagudtryk eller begreb bliver benyttet, introduceres det enten i en fodnote eller i den kontekstuelle sammenhæng. \\

{\bf Meta-tekst}\\
Hvert afsnit indledes med en meta-tekst som er anført i kursiv. \\

\textit{Eksempel: Dette afsnit indeholder...}. \\

\textbf{Opsummeringer}\\
Slutningen af ethvert afsnit følges af en opsummering. Dette er opstillet i nogle få og præcise punkter, som beskriver, hvad der kan konkluderes ud fra afsnittet.\\

\textit{Eksempel:
\begin{itemize_small}
    \item Det er vigtigt at...
    \item Det er ikke muligt at...
\end{itemize_small}}

{\bf Tabeller}\\
Der vil løbende blive fremvist tabeller, som anvendts til at give læseren en bedre visualisering i kontekst med tekstens indhold.

\begin{table}[h]
    \centering
    \begin{tabular}{ l | l }
        \textbf{Overskrift} & \textbf{Overskrift} \\
        \hline \hline
        A & B \\
    \end{tabular}
    \caption{\textit{Eksempel på en tabel. \tabelgroup}}
    \label{tab:abc}
\end{table}

{\bf Kodeeksempler}\\
Der vil i problemløsningensafsnittet løbende blive refereret til kodeeksempler.\\
\textit{Eksempel:}

\CSharp{Kode/eksempel.cs}{Beskrivelsen af den viste kode er placeret her}{Code_example}
\input{Kapitler/Indledende/Indholdsfortegnelse.tex}

\chapter{Indledning}
%Fjerde forsøg - Skrevet af Jonas d. 03-03-2014
%Rettet af Jimmi. d. 03-03-2014 kl. 12.34
%Rettet af Søren. d. 09-03-2014 kl. 15.12
%Rettet af Rune 11-03-14 kl. 12:45
%Rettet af Jimmi. d. 13-03-2014 kl. 21.24
% Rune 20-05-2014 09:30

\label{sec:indledning}
I Danmark er der registreret omkring 101.000 organisationer, hvoraf de 18.000 af dem er sportsforeninger af forskellig art. En forening er en samling af mennesker, der er organiseret efter en fælles interesse, og kan inddeles efter flere kriterier såsom struktur, formål og geografisk placering. \cite{Antal_Frivillige}\cite{DefinitionForening} Fælles for alle typer af foreninger er, at selve foreningen er en tidskrævende beskæftigelse - både i forhold til den aktivitet, som foreningen omfatter, og i forhold til det administrative arbejde, der er påkrævet for at drive en forening.\\

Umiddelbart ville det administrative arbejde ikke være et problem, hvis personerne der varetog disse opgaver, var aflønnede. Langt de fleste foreninger er dog drevet af frivillige mennesker, som har et fuldtidsarbejde ved siden af. Dette medvirker til, at arbejdsopgaverne ofte fordeles jævnbyrdigt på foreningens bestyrelsesposter for at kunne efterkomme de tidsmæssige begrænsninger.\\

Det administrative arbejde består bl.a. i, at foreningen årligt skal kunne dokumentere økonomiske transaktioner, medlemsoplysninger og andre bogføringsdokumenter, hvor mange foreninger også har en række resourcer til rådighed, som ligeledes skal administreres. Disse resourcer kan afhængigt af foreningens type bestå af forskellige områder og/eller materiel, som foreningen ejer eller bestyrer, og ønsker at stille til rådighed for deres medlemmer. Dette giver en del arbejde, når resourcerne forventes at blive ligeligt fordelt, og når medlemmerne skal stilles til ansvar for resourcerne, som de bruger.\\

Kigger man på en bestemt type forening, nærmere bestemt ro- og sejlklubber, har de nogle helt konkrete problemstillinger i forhold til udlån og administration af både. Klubben har materiel i form af både og lokaler, som kan benyttes af medlemmerne. Især i forbindelse med bådudlån, er der flere administrative opgaver, som skal varetages. Udlån skal registeres, så ansvaret for båden bliver placeret hos de rigtige, beskadigede både skal markeres og repareres, og ture skal registreres, så der er et overblik over hvem, der er på vandet. \\

Alle de førnævnte arbejdsopgaver er noget, man kunne forestille sig, at et IT-system kunne afhjælpe. Hvis foreningen stoler på dets medlemmer, er der ikke noget i vejen for, at medlemmerne selv kunne varetage de førnævnte opgaver i forhold til registrering og markering, hvis blot de fik hjælp af et IT-system, som på samme tid kunne give et overblik over materiel.\\

På baggrund af denne indledning, kan følgende initierende problemstilling dermed opstilles:\\

\textit{Hvordan kan et IT-system hjælpe ro- og seljklubber med at administrere materiel og medlemmer?}

% Skrevet,  Mikkel  03-03-2014
% Rettet,   Rune    08-04-2014
% Rettet af Mikkel 24-04-2014
% Rune 18-05-2014 15:50
% Jimmi 19-05-2014 15:41

\section{Rapportens struktur}


\textbf{Problemanalyse}\\
Dette kapitel har til formål at belyse Aalborg Roklub’ eksisterende IT-systemer, hvorved fejl, mangler og problemstillinger kan forefindes. De tilgængelige IT-systemer på markedet bliver også undersøgt, for at kunne afgøre om hvorvidt disse kan løse Aalborg Roklubs mangler og fejl. Hertil findes følgende afsnit:
    \begin{itemize}
    \item  \textbf{Dataindsamling}: I udarbejdelsen af projektet blev Aalborg Roklub interviewet. Overvejelser, resultater og beskrivelse af klubben findes i dette afsnit

    \item \textbf{Teknologianalyse}: Har til formål at belyse kravene i teknologien, finde eksisterende teknologi og derefter udføre tests på disse med fokus på brugervenlighed

    \item \textbf{Interessentanalyse}: Anvendes til at analysere forskellige interessenter for derefter at finde ud af, hvilke målgrupper der kunne have interesse i produktet
    \end{itemize}

\textbf{Problembeskrivelse}\\
Kapitlet indeholder en opsamling af alle foregående afgrænsninger, som er blevet foretaget undervejs i problemanalysen. Disse afgrænsninger bliver afslutningsvis konkretiseret til en problemformulering. Problemformuleringen danner grundlaget for problemløsningen. \\

\textbf{Problemløsning}\\
I dette kapitel behandles den opstillede problemformulering. 

    \begin{itemize}
    \item \textbf{Kravspecifikationer}: Dette afsnit gennemgår en kort overgang fra problembeskrivelsens abstrakte problemformulering til et række funktionelle og non-funktionelle krav til produktløsningen
    \item  \textbf{Teori}: Afsnittet berører de teoretiske aspekter indenfor design og databaser, som bruges til at afgøre hvordan disse skal konstrueres mest hensigtsmæssigt
    \item \textbf{Metode}: Afsnittet redegør for de udviklingsmetoder og redskaber, som er anvendt under projektudarbejdelsen
    \item \textbf{Implementering}: Afsnittet viser dele af projektets endelige løsning, hvordan disse er implementeret, og om de har opfyldt de opstillede krav til produktet
    \end{itemize}


\textbf{Diskussion}\\
Afsnittet tager projektets afgrænsnings- og funktionalitetvalg op til diskussion, for dermed at redegøre for, om der blev foretaget det korrekte valg. \\

% Rune 20-05-2014 09:35
\textbf{Konklusion}\\
I konklusionen foretages der en opsamling af gruppens endelige produkt og en belysning på, om hvorvidt produktet løser de opstillede kravspecifikationer. \\

\textbf{Perspektivering}\\
Perspektiveringen opstiller interessante udvidelses-potentialer og om hvordan, at programmet kan blive tilpasset, så det kan bruges til andre klubber eller foreninger end blot Aalborg Roklub.



\clearpage
\section*{Forord}
% Rune 18-05-2014 14:40
% Rune 20-05-2014 09:20
Denne rapport er udarbejdet af en gruppe software-studerende på 2. semester på Det Teknisk-Naturvidenskabelig Fakultet på Aalborg Universitet.\\

I projektet har gruppen taget udgangspunkt i AAU-modellen, som tager afsæt i problemorienteret projekt- og gruppeorganiseret læring, hvilket grundlæggende består i en læringsproces, der indledes med en problemanalyse. Denne analyse danner herefter grundlag for rapportens videre bearbejdning af problemets løsning.
\section*{Læsevejledning}
% Rune 18-05-2014 14:50
Følgende læsevejledning og det foregående forord er inspireret af P1-projektet "Automatisér dit hjem" \cite{p1_projekt}.\\

{\bf Kildehenvisning} \\
Ved kildehenvisninger anvendes Vancouver-metoden. Dette bliver brugt ud fra hvert afsnit eller citat, som henviser til en specifik kilde i litteraturlisten. Her vil der stå et tal indkapslet i hårde parenteser. \\

\textit{Eksempel: Påstand/citat\emph{[1]}}.\\

Hvis der i et afsnit er anvendt store mængder information fra en kilde, står kilden efter punktummet i afslutningen af afsnittet.\\

\textit{Eksempel: Afsnit 1}\\

Litteraturlisten er kompositorisk opsat således, at kilderne er placeret efter den numeriske orden, de er anvendt i teksten. Ud fra deres repræsentative nummerering er alle relevante informationer associeret. \\

\textit{Eksempel: Forfatter(e), Titel på artikel/afsnit, sider med relevant information, bogens titel, redaktør, forlag, udgivelsesårs og ISBN-nummer.} \\

Hvis nogle af informationerne mangler, som f.eks. forfatterens navn, udelades disse informationer i kildebeskrivelsen. \\

{\bf Figurhenvisning} \\
I rapporten vil der løbende blive refereret til figurer eller illustrationer. I den kontekstuelle sammenhæng, hvor figurerne anvendes, vil dette være angivet på følgende måde: \textit{Afsnit.nummer} \\

\textit{Eksempel: 2. figur i 3. afsnit vil være angivet med følgende referencenummer: 3.2}. \\

Under figuren vil figurenes referencenummer samt en dertilhørende figurbeskrivelse, som forklarer figurens relevans være påført den pågældende figur. \\ 

\textit{Eksempel: "Erklæring som en figur er kildemateriale til"}. Se figur \ref{fig:FigurEksempel}.\\
\figur{Figurer/Figureksempel.png}{Eksempel på figur beskrivelse.}{FigurEksempel}{0.3}

{\bf Fodnote henvisning}\\
Fodnoter benyttes til at inkludere yderligere informationer om et specifikt begreb, fagudtryk eller forkortelse. Fodnoterne beskriver sjældent vitale informationer for forståelsen af det omtalte. En fodnote ses ved et lille tal som henviser til sit modstykke i bunden af siden, hvor den vedhæftede tekst er vist. \\

\textit{Eksempel: Fodnote\footnote{Eksempel på fodnote}}.\\

Første gang et specielt fagudtryk eller begreb bliver benyttet, introduceres det enten i en fodnote eller i den kontekstuelle sammenhæng. \\

{\bf Meta-tekst}\\
Hvert afsnit indledes med en meta-tekst som er anført i kursiv. \\

\textit{Eksempel: Dette afsnit indeholder...}. \\

\textbf{Opsummeringer}\\
Slutningen af ethvert afsnit følges af en opsummering. Dette er opstillet i nogle få og præcise punkter, som beskriver, hvad der kan konkluderes ud fra afsnittet.\\

\textit{Eksempel:
\begin{itemize_small}
    \item Det er vigtigt at...
    \item Det er ikke muligt at...
\end{itemize_small}}

{\bf Tabeller}\\
Der vil løbende blive fremvist tabeller, som anvendts til at give læseren en bedre visualisering i kontekst med tekstens indhold.

\begin{table}[h]
    \centering
    \begin{tabular}{ l | l }
        \textbf{Overskrift} & \textbf{Overskrift} \\
        \hline \hline
        A & B \\
    \end{tabular}
    \caption{\textit{Eksempel på en tabel. \tabelgroup}}
    \label{tab:abc}
\end{table}

{\bf Kodeeksempler}\\
Der vil i problemløsningensafsnittet løbende blive refereret til kodeeksempler.\\
\textit{Eksempel:}

\CSharp{Kode/eksempel.cs}{Beskrivelsen af den viste kode er placeret her}{Code_example}
\input{Kapitler/Indledende/Indholdsfortegnelse.tex}

\chapter{Indledning}
%Fjerde forsøg - Skrevet af Jonas d. 03-03-2014
%Rettet af Jimmi. d. 03-03-2014 kl. 12.34
%Rettet af Søren. d. 09-03-2014 kl. 15.12
%Rettet af Rune 11-03-14 kl. 12:45
%Rettet af Jimmi. d. 13-03-2014 kl. 21.24
% Rune 20-05-2014 09:30

\label{sec:indledning}
I Danmark er der registreret omkring 101.000 organisationer, hvoraf de 18.000 af dem er sportsforeninger af forskellig art. En forening er en samling af mennesker, der er organiseret efter en fælles interesse, og kan inddeles efter flere kriterier såsom struktur, formål og geografisk placering. \cite{Antal_Frivillige}\cite{DefinitionForening} Fælles for alle typer af foreninger er, at selve foreningen er en tidskrævende beskæftigelse - både i forhold til den aktivitet, som foreningen omfatter, og i forhold til det administrative arbejde, der er påkrævet for at drive en forening.\\

Umiddelbart ville det administrative arbejde ikke være et problem, hvis personerne der varetog disse opgaver, var aflønnede. Langt de fleste foreninger er dog drevet af frivillige mennesker, som har et fuldtidsarbejde ved siden af. Dette medvirker til, at arbejdsopgaverne ofte fordeles jævnbyrdigt på foreningens bestyrelsesposter for at kunne efterkomme de tidsmæssige begrænsninger.\\

Det administrative arbejde består bl.a. i, at foreningen årligt skal kunne dokumentere økonomiske transaktioner, medlemsoplysninger og andre bogføringsdokumenter, hvor mange foreninger også har en række resourcer til rådighed, som ligeledes skal administreres. Disse resourcer kan afhængigt af foreningens type bestå af forskellige områder og/eller materiel, som foreningen ejer eller bestyrer, og ønsker at stille til rådighed for deres medlemmer. Dette giver en del arbejde, når resourcerne forventes at blive ligeligt fordelt, og når medlemmerne skal stilles til ansvar for resourcerne, som de bruger.\\

Kigger man på en bestemt type forening, nærmere bestemt ro- og sejlklubber, har de nogle helt konkrete problemstillinger i forhold til udlån og administration af både. Klubben har materiel i form af både og lokaler, som kan benyttes af medlemmerne. Især i forbindelse med bådudlån, er der flere administrative opgaver, som skal varetages. Udlån skal registeres, så ansvaret for båden bliver placeret hos de rigtige, beskadigede både skal markeres og repareres, og ture skal registreres, så der er et overblik over hvem, der er på vandet. \\

Alle de førnævnte arbejdsopgaver er noget, man kunne forestille sig, at et IT-system kunne afhjælpe. Hvis foreningen stoler på dets medlemmer, er der ikke noget i vejen for, at medlemmerne selv kunne varetage de førnævnte opgaver i forhold til registrering og markering, hvis blot de fik hjælp af et IT-system, som på samme tid kunne give et overblik over materiel.\\

På baggrund af denne indledning, kan følgende initierende problemstilling dermed opstilles:\\

\textit{Hvordan kan et IT-system hjælpe ro- og seljklubber med at administrere materiel og medlemmer?}

% Skrevet,  Mikkel  03-03-2014
% Rettet,   Rune    08-04-2014
% Rettet af Mikkel 24-04-2014
% Rune 18-05-2014 15:50
% Jimmi 19-05-2014 15:41

\section{Rapportens struktur}


\textbf{Problemanalyse}\\
Dette kapitel har til formål at belyse Aalborg Roklub’ eksisterende IT-systemer, hvorved fejl, mangler og problemstillinger kan forefindes. De tilgængelige IT-systemer på markedet bliver også undersøgt, for at kunne afgøre om hvorvidt disse kan løse Aalborg Roklubs mangler og fejl. Hertil findes følgende afsnit:
    \begin{itemize}
    \item  \textbf{Dataindsamling}: I udarbejdelsen af projektet blev Aalborg Roklub interviewet. Overvejelser, resultater og beskrivelse af klubben findes i dette afsnit

    \item \textbf{Teknologianalyse}: Har til formål at belyse kravene i teknologien, finde eksisterende teknologi og derefter udføre tests på disse med fokus på brugervenlighed

    \item \textbf{Interessentanalyse}: Anvendes til at analysere forskellige interessenter for derefter at finde ud af, hvilke målgrupper der kunne have interesse i produktet
    \end{itemize}

\textbf{Problembeskrivelse}\\
Kapitlet indeholder en opsamling af alle foregående afgrænsninger, som er blevet foretaget undervejs i problemanalysen. Disse afgrænsninger bliver afslutningsvis konkretiseret til en problemformulering. Problemformuleringen danner grundlaget for problemløsningen. \\

\textbf{Problemløsning}\\
I dette kapitel behandles den opstillede problemformulering. 

    \begin{itemize}
    \item \textbf{Kravspecifikationer}: Dette afsnit gennemgår en kort overgang fra problembeskrivelsens abstrakte problemformulering til et række funktionelle og non-funktionelle krav til produktløsningen
    \item  \textbf{Teori}: Afsnittet berører de teoretiske aspekter indenfor design og databaser, som bruges til at afgøre hvordan disse skal konstrueres mest hensigtsmæssigt
    \item \textbf{Metode}: Afsnittet redegør for de udviklingsmetoder og redskaber, som er anvendt under projektudarbejdelsen
    \item \textbf{Implementering}: Afsnittet viser dele af projektets endelige løsning, hvordan disse er implementeret, og om de har opfyldt de opstillede krav til produktet
    \end{itemize}


\textbf{Diskussion}\\
Afsnittet tager projektets afgrænsnings- og funktionalitetvalg op til diskussion, for dermed at redegøre for, om der blev foretaget det korrekte valg. \\

% Rune 20-05-2014 09:35
\textbf{Konklusion}\\
I konklusionen foretages der en opsamling af gruppens endelige produkt og en belysning på, om hvorvidt produktet løser de opstillede kravspecifikationer. \\

\textbf{Perspektivering}\\
Perspektiveringen opstiller interessante udvidelses-potentialer og om hvordan, at programmet kan blive tilpasset, så det kan bruges til andre klubber eller foreninger end blot Aalborg Roklub.



\clearpage
\fi

% Source code
\ifdefined\CMDCode
    \input{Kapitler/Indledende/Titelblad/Index.tex}
\input{Kapitler/Indledende/Forord.tex}
\input{Kapitler/Indledende/Lasevejledning.tex}
\input{Kapitler/Indledende/Indholdsfortegnelse.tex}

\chapter{Indledning}
\input{Kapitler/Indledende/Indledning.tex}
\input{Kapitler/Indledende/Struktur.tex}


\clearpage
\section*{Forord}
% Rune 18-05-2014 14:40
% Rune 20-05-2014 09:20
Denne rapport er udarbejdet af en gruppe software-studerende på 2. semester på Det Teknisk-Naturvidenskabelig Fakultet på Aalborg Universitet.\\

I projektet har gruppen taget udgangspunkt i AAU-modellen, som tager afsæt i problemorienteret projekt- og gruppeorganiseret læring, hvilket grundlæggende består i en læringsproces, der indledes med en problemanalyse. Denne analyse danner herefter grundlag for rapportens videre bearbejdning af problemets løsning.
\section*{Læsevejledning}
% Rune 18-05-2014 14:50
Følgende læsevejledning og det foregående forord er inspireret af P1-projektet "Automatisér dit hjem" \cite{p1_projekt}.\\

{\bf Kildehenvisning} \\
Ved kildehenvisninger anvendes Vancouver-metoden. Dette bliver brugt ud fra hvert afsnit eller citat, som henviser til en specifik kilde i litteraturlisten. Her vil der stå et tal indkapslet i hårde parenteser. \\

\textit{Eksempel: Påstand/citat\emph{[1]}}.\\

Hvis der i et afsnit er anvendt store mængder information fra en kilde, står kilden efter punktummet i afslutningen af afsnittet.\\

\textit{Eksempel: Afsnit 1}\\

Litteraturlisten er kompositorisk opsat således, at kilderne er placeret efter den numeriske orden, de er anvendt i teksten. Ud fra deres repræsentative nummerering er alle relevante informationer associeret. \\

\textit{Eksempel: Forfatter(e), Titel på artikel/afsnit, sider med relevant information, bogens titel, redaktør, forlag, udgivelsesårs og ISBN-nummer.} \\

Hvis nogle af informationerne mangler, som f.eks. forfatterens navn, udelades disse informationer i kildebeskrivelsen. \\

{\bf Figurhenvisning} \\
I rapporten vil der løbende blive refereret til figurer eller illustrationer. I den kontekstuelle sammenhæng, hvor figurerne anvendes, vil dette være angivet på følgende måde: \textit{Afsnit.nummer} \\

\textit{Eksempel: 2. figur i 3. afsnit vil være angivet med følgende referencenummer: 3.2}. \\

Under figuren vil figurenes referencenummer samt en dertilhørende figurbeskrivelse, som forklarer figurens relevans være påført den pågældende figur. \\ 

\textit{Eksempel: "Erklæring som en figur er kildemateriale til"}. Se figur \ref{fig:FigurEksempel}.\\
\figur{Figurer/Figureksempel.png}{Eksempel på figur beskrivelse.}{FigurEksempel}{0.3}

{\bf Fodnote henvisning}\\
Fodnoter benyttes til at inkludere yderligere informationer om et specifikt begreb, fagudtryk eller forkortelse. Fodnoterne beskriver sjældent vitale informationer for forståelsen af det omtalte. En fodnote ses ved et lille tal som henviser til sit modstykke i bunden af siden, hvor den vedhæftede tekst er vist. \\

\textit{Eksempel: Fodnote\footnote{Eksempel på fodnote}}.\\

Første gang et specielt fagudtryk eller begreb bliver benyttet, introduceres det enten i en fodnote eller i den kontekstuelle sammenhæng. \\

{\bf Meta-tekst}\\
Hvert afsnit indledes med en meta-tekst som er anført i kursiv. \\

\textit{Eksempel: Dette afsnit indeholder...}. \\

\textbf{Opsummeringer}\\
Slutningen af ethvert afsnit følges af en opsummering. Dette er opstillet i nogle få og præcise punkter, som beskriver, hvad der kan konkluderes ud fra afsnittet.\\

\textit{Eksempel:
\begin{itemize_small}
    \item Det er vigtigt at...
    \item Det er ikke muligt at...
\end{itemize_small}}

{\bf Tabeller}\\
Der vil løbende blive fremvist tabeller, som anvendts til at give læseren en bedre visualisering i kontekst med tekstens indhold.

\begin{table}[h]
    \centering
    \begin{tabular}{ l | l }
        \textbf{Overskrift} & \textbf{Overskrift} \\
        \hline \hline
        A & B \\
    \end{tabular}
    \caption{\textit{Eksempel på en tabel. \tabelgroup}}
    \label{tab:abc}
\end{table}

{\bf Kodeeksempler}\\
Der vil i problemløsningensafsnittet løbende blive refereret til kodeeksempler.\\
\textit{Eksempel:}

\CSharp{Kode/eksempel.cs}{Beskrivelsen af den viste kode er placeret her}{Code_example}
\input{Kapitler/Indledende/Indholdsfortegnelse.tex}

\chapter{Indledning}
%Fjerde forsøg - Skrevet af Jonas d. 03-03-2014
%Rettet af Jimmi. d. 03-03-2014 kl. 12.34
%Rettet af Søren. d. 09-03-2014 kl. 15.12
%Rettet af Rune 11-03-14 kl. 12:45
%Rettet af Jimmi. d. 13-03-2014 kl. 21.24
% Rune 20-05-2014 09:30

\label{sec:indledning}
I Danmark er der registreret omkring 101.000 organisationer, hvoraf de 18.000 af dem er sportsforeninger af forskellig art. En forening er en samling af mennesker, der er organiseret efter en fælles interesse, og kan inddeles efter flere kriterier såsom struktur, formål og geografisk placering. \cite{Antal_Frivillige}\cite{DefinitionForening} Fælles for alle typer af foreninger er, at selve foreningen er en tidskrævende beskæftigelse - både i forhold til den aktivitet, som foreningen omfatter, og i forhold til det administrative arbejde, der er påkrævet for at drive en forening.\\

Umiddelbart ville det administrative arbejde ikke være et problem, hvis personerne der varetog disse opgaver, var aflønnede. Langt de fleste foreninger er dog drevet af frivillige mennesker, som har et fuldtidsarbejde ved siden af. Dette medvirker til, at arbejdsopgaverne ofte fordeles jævnbyrdigt på foreningens bestyrelsesposter for at kunne efterkomme de tidsmæssige begrænsninger.\\

Det administrative arbejde består bl.a. i, at foreningen årligt skal kunne dokumentere økonomiske transaktioner, medlemsoplysninger og andre bogføringsdokumenter, hvor mange foreninger også har en række resourcer til rådighed, som ligeledes skal administreres. Disse resourcer kan afhængigt af foreningens type bestå af forskellige områder og/eller materiel, som foreningen ejer eller bestyrer, og ønsker at stille til rådighed for deres medlemmer. Dette giver en del arbejde, når resourcerne forventes at blive ligeligt fordelt, og når medlemmerne skal stilles til ansvar for resourcerne, som de bruger.\\

Kigger man på en bestemt type forening, nærmere bestemt ro- og sejlklubber, har de nogle helt konkrete problemstillinger i forhold til udlån og administration af både. Klubben har materiel i form af både og lokaler, som kan benyttes af medlemmerne. Især i forbindelse med bådudlån, er der flere administrative opgaver, som skal varetages. Udlån skal registeres, så ansvaret for båden bliver placeret hos de rigtige, beskadigede både skal markeres og repareres, og ture skal registreres, så der er et overblik over hvem, der er på vandet. \\

Alle de førnævnte arbejdsopgaver er noget, man kunne forestille sig, at et IT-system kunne afhjælpe. Hvis foreningen stoler på dets medlemmer, er der ikke noget i vejen for, at medlemmerne selv kunne varetage de førnævnte opgaver i forhold til registrering og markering, hvis blot de fik hjælp af et IT-system, som på samme tid kunne give et overblik over materiel.\\

På baggrund af denne indledning, kan følgende initierende problemstilling dermed opstilles:\\

\textit{Hvordan kan et IT-system hjælpe ro- og seljklubber med at administrere materiel og medlemmer?}

% Skrevet,  Mikkel  03-03-2014
% Rettet,   Rune    08-04-2014
% Rettet af Mikkel 24-04-2014
% Rune 18-05-2014 15:50
% Jimmi 19-05-2014 15:41

\section{Rapportens struktur}


\textbf{Problemanalyse}\\
Dette kapitel har til formål at belyse Aalborg Roklub’ eksisterende IT-systemer, hvorved fejl, mangler og problemstillinger kan forefindes. De tilgængelige IT-systemer på markedet bliver også undersøgt, for at kunne afgøre om hvorvidt disse kan løse Aalborg Roklubs mangler og fejl. Hertil findes følgende afsnit:
    \begin{itemize}
    \item  \textbf{Dataindsamling}: I udarbejdelsen af projektet blev Aalborg Roklub interviewet. Overvejelser, resultater og beskrivelse af klubben findes i dette afsnit

    \item \textbf{Teknologianalyse}: Har til formål at belyse kravene i teknologien, finde eksisterende teknologi og derefter udføre tests på disse med fokus på brugervenlighed

    \item \textbf{Interessentanalyse}: Anvendes til at analysere forskellige interessenter for derefter at finde ud af, hvilke målgrupper der kunne have interesse i produktet
    \end{itemize}

\textbf{Problembeskrivelse}\\
Kapitlet indeholder en opsamling af alle foregående afgrænsninger, som er blevet foretaget undervejs i problemanalysen. Disse afgrænsninger bliver afslutningsvis konkretiseret til en problemformulering. Problemformuleringen danner grundlaget for problemløsningen. \\

\textbf{Problemløsning}\\
I dette kapitel behandles den opstillede problemformulering. 

    \begin{itemize}
    \item \textbf{Kravspecifikationer}: Dette afsnit gennemgår en kort overgang fra problembeskrivelsens abstrakte problemformulering til et række funktionelle og non-funktionelle krav til produktløsningen
    \item  \textbf{Teori}: Afsnittet berører de teoretiske aspekter indenfor design og databaser, som bruges til at afgøre hvordan disse skal konstrueres mest hensigtsmæssigt
    \item \textbf{Metode}: Afsnittet redegør for de udviklingsmetoder og redskaber, som er anvendt under projektudarbejdelsen
    \item \textbf{Implementering}: Afsnittet viser dele af projektets endelige løsning, hvordan disse er implementeret, og om de har opfyldt de opstillede krav til produktet
    \end{itemize}


\textbf{Diskussion}\\
Afsnittet tager projektets afgrænsnings- og funktionalitetvalg op til diskussion, for dermed at redegøre for, om der blev foretaget det korrekte valg. \\

% Rune 20-05-2014 09:35
\textbf{Konklusion}\\
I konklusionen foretages der en opsamling af gruppens endelige produkt og en belysning på, om hvorvidt produktet løser de opstillede kravspecifikationer. \\

\textbf{Perspektivering}\\
Perspektiveringen opstiller interessante udvidelses-potentialer og om hvordan, at programmet kan blive tilpasset, så det kan bruges til andre klubber eller foreninger end blot Aalborg Roklub.



\clearpage
\section*{Forord}
% Rune 18-05-2014 14:40
% Rune 20-05-2014 09:20
Denne rapport er udarbejdet af en gruppe software-studerende på 2. semester på Det Teknisk-Naturvidenskabelig Fakultet på Aalborg Universitet.\\

I projektet har gruppen taget udgangspunkt i AAU-modellen, som tager afsæt i problemorienteret projekt- og gruppeorganiseret læring, hvilket grundlæggende består i en læringsproces, der indledes med en problemanalyse. Denne analyse danner herefter grundlag for rapportens videre bearbejdning af problemets løsning.
\section*{Læsevejledning}
% Rune 18-05-2014 14:50
Følgende læsevejledning og det foregående forord er inspireret af P1-projektet "Automatisér dit hjem" \cite{p1_projekt}.\\

{\bf Kildehenvisning} \\
Ved kildehenvisninger anvendes Vancouver-metoden. Dette bliver brugt ud fra hvert afsnit eller citat, som henviser til en specifik kilde i litteraturlisten. Her vil der stå et tal indkapslet i hårde parenteser. \\

\textit{Eksempel: Påstand/citat\emph{[1]}}.\\

Hvis der i et afsnit er anvendt store mængder information fra en kilde, står kilden efter punktummet i afslutningen af afsnittet.\\

\textit{Eksempel: Afsnit 1}\\

Litteraturlisten er kompositorisk opsat således, at kilderne er placeret efter den numeriske orden, de er anvendt i teksten. Ud fra deres repræsentative nummerering er alle relevante informationer associeret. \\

\textit{Eksempel: Forfatter(e), Titel på artikel/afsnit, sider med relevant information, bogens titel, redaktør, forlag, udgivelsesårs og ISBN-nummer.} \\

Hvis nogle af informationerne mangler, som f.eks. forfatterens navn, udelades disse informationer i kildebeskrivelsen. \\

{\bf Figurhenvisning} \\
I rapporten vil der løbende blive refereret til figurer eller illustrationer. I den kontekstuelle sammenhæng, hvor figurerne anvendes, vil dette være angivet på følgende måde: \textit{Afsnit.nummer} \\

\textit{Eksempel: 2. figur i 3. afsnit vil være angivet med følgende referencenummer: 3.2}. \\

Under figuren vil figurenes referencenummer samt en dertilhørende figurbeskrivelse, som forklarer figurens relevans være påført den pågældende figur. \\ 

\textit{Eksempel: "Erklæring som en figur er kildemateriale til"}. Se figur \ref{fig:FigurEksempel}.\\
\figur{Figurer/Figureksempel.png}{Eksempel på figur beskrivelse.}{FigurEksempel}{0.3}

{\bf Fodnote henvisning}\\
Fodnoter benyttes til at inkludere yderligere informationer om et specifikt begreb, fagudtryk eller forkortelse. Fodnoterne beskriver sjældent vitale informationer for forståelsen af det omtalte. En fodnote ses ved et lille tal som henviser til sit modstykke i bunden af siden, hvor den vedhæftede tekst er vist. \\

\textit{Eksempel: Fodnote\footnote{Eksempel på fodnote}}.\\

Første gang et specielt fagudtryk eller begreb bliver benyttet, introduceres det enten i en fodnote eller i den kontekstuelle sammenhæng. \\

{\bf Meta-tekst}\\
Hvert afsnit indledes med en meta-tekst som er anført i kursiv. \\

\textit{Eksempel: Dette afsnit indeholder...}. \\

\textbf{Opsummeringer}\\
Slutningen af ethvert afsnit følges af en opsummering. Dette er opstillet i nogle få og præcise punkter, som beskriver, hvad der kan konkluderes ud fra afsnittet.\\

\textit{Eksempel:
\begin{itemize_small}
    \item Det er vigtigt at...
    \item Det er ikke muligt at...
\end{itemize_small}}

{\bf Tabeller}\\
Der vil løbende blive fremvist tabeller, som anvendts til at give læseren en bedre visualisering i kontekst med tekstens indhold.

\begin{table}[h]
    \centering
    \begin{tabular}{ l | l }
        \textbf{Overskrift} & \textbf{Overskrift} \\
        \hline \hline
        A & B \\
    \end{tabular}
    \caption{\textit{Eksempel på en tabel. \tabelgroup}}
    \label{tab:abc}
\end{table}

{\bf Kodeeksempler}\\
Der vil i problemløsningensafsnittet løbende blive refereret til kodeeksempler.\\
\textit{Eksempel:}

\CSharp{Kode/eksempel.cs}{Beskrivelsen af den viste kode er placeret her}{Code_example}
\input{Kapitler/Indledende/Indholdsfortegnelse.tex}

\chapter{Indledning}
%Fjerde forsøg - Skrevet af Jonas d. 03-03-2014
%Rettet af Jimmi. d. 03-03-2014 kl. 12.34
%Rettet af Søren. d. 09-03-2014 kl. 15.12
%Rettet af Rune 11-03-14 kl. 12:45
%Rettet af Jimmi. d. 13-03-2014 kl. 21.24
% Rune 20-05-2014 09:30

\label{sec:indledning}
I Danmark er der registreret omkring 101.000 organisationer, hvoraf de 18.000 af dem er sportsforeninger af forskellig art. En forening er en samling af mennesker, der er organiseret efter en fælles interesse, og kan inddeles efter flere kriterier såsom struktur, formål og geografisk placering. \cite{Antal_Frivillige}\cite{DefinitionForening} Fælles for alle typer af foreninger er, at selve foreningen er en tidskrævende beskæftigelse - både i forhold til den aktivitet, som foreningen omfatter, og i forhold til det administrative arbejde, der er påkrævet for at drive en forening.\\

Umiddelbart ville det administrative arbejde ikke være et problem, hvis personerne der varetog disse opgaver, var aflønnede. Langt de fleste foreninger er dog drevet af frivillige mennesker, som har et fuldtidsarbejde ved siden af. Dette medvirker til, at arbejdsopgaverne ofte fordeles jævnbyrdigt på foreningens bestyrelsesposter for at kunne efterkomme de tidsmæssige begrænsninger.\\

Det administrative arbejde består bl.a. i, at foreningen årligt skal kunne dokumentere økonomiske transaktioner, medlemsoplysninger og andre bogføringsdokumenter, hvor mange foreninger også har en række resourcer til rådighed, som ligeledes skal administreres. Disse resourcer kan afhængigt af foreningens type bestå af forskellige områder og/eller materiel, som foreningen ejer eller bestyrer, og ønsker at stille til rådighed for deres medlemmer. Dette giver en del arbejde, når resourcerne forventes at blive ligeligt fordelt, og når medlemmerne skal stilles til ansvar for resourcerne, som de bruger.\\

Kigger man på en bestemt type forening, nærmere bestemt ro- og sejlklubber, har de nogle helt konkrete problemstillinger i forhold til udlån og administration af både. Klubben har materiel i form af både og lokaler, som kan benyttes af medlemmerne. Især i forbindelse med bådudlån, er der flere administrative opgaver, som skal varetages. Udlån skal registeres, så ansvaret for båden bliver placeret hos de rigtige, beskadigede både skal markeres og repareres, og ture skal registreres, så der er et overblik over hvem, der er på vandet. \\

Alle de førnævnte arbejdsopgaver er noget, man kunne forestille sig, at et IT-system kunne afhjælpe. Hvis foreningen stoler på dets medlemmer, er der ikke noget i vejen for, at medlemmerne selv kunne varetage de førnævnte opgaver i forhold til registrering og markering, hvis blot de fik hjælp af et IT-system, som på samme tid kunne give et overblik over materiel.\\

På baggrund af denne indledning, kan følgende initierende problemstilling dermed opstilles:\\

\textit{Hvordan kan et IT-system hjælpe ro- og seljklubber med at administrere materiel og medlemmer?}

% Skrevet,  Mikkel  03-03-2014
% Rettet,   Rune    08-04-2014
% Rettet af Mikkel 24-04-2014
% Rune 18-05-2014 15:50
% Jimmi 19-05-2014 15:41

\section{Rapportens struktur}


\textbf{Problemanalyse}\\
Dette kapitel har til formål at belyse Aalborg Roklub’ eksisterende IT-systemer, hvorved fejl, mangler og problemstillinger kan forefindes. De tilgængelige IT-systemer på markedet bliver også undersøgt, for at kunne afgøre om hvorvidt disse kan løse Aalborg Roklubs mangler og fejl. Hertil findes følgende afsnit:
    \begin{itemize}
    \item  \textbf{Dataindsamling}: I udarbejdelsen af projektet blev Aalborg Roklub interviewet. Overvejelser, resultater og beskrivelse af klubben findes i dette afsnit

    \item \textbf{Teknologianalyse}: Har til formål at belyse kravene i teknologien, finde eksisterende teknologi og derefter udføre tests på disse med fokus på brugervenlighed

    \item \textbf{Interessentanalyse}: Anvendes til at analysere forskellige interessenter for derefter at finde ud af, hvilke målgrupper der kunne have interesse i produktet
    \end{itemize}

\textbf{Problembeskrivelse}\\
Kapitlet indeholder en opsamling af alle foregående afgrænsninger, som er blevet foretaget undervejs i problemanalysen. Disse afgrænsninger bliver afslutningsvis konkretiseret til en problemformulering. Problemformuleringen danner grundlaget for problemløsningen. \\

\textbf{Problemløsning}\\
I dette kapitel behandles den opstillede problemformulering. 

    \begin{itemize}
    \item \textbf{Kravspecifikationer}: Dette afsnit gennemgår en kort overgang fra problembeskrivelsens abstrakte problemformulering til et række funktionelle og non-funktionelle krav til produktløsningen
    \item  \textbf{Teori}: Afsnittet berører de teoretiske aspekter indenfor design og databaser, som bruges til at afgøre hvordan disse skal konstrueres mest hensigtsmæssigt
    \item \textbf{Metode}: Afsnittet redegør for de udviklingsmetoder og redskaber, som er anvendt under projektudarbejdelsen
    \item \textbf{Implementering}: Afsnittet viser dele af projektets endelige løsning, hvordan disse er implementeret, og om de har opfyldt de opstillede krav til produktet
    \end{itemize}


\textbf{Diskussion}\\
Afsnittet tager projektets afgrænsnings- og funktionalitetvalg op til diskussion, for dermed at redegøre for, om der blev foretaget det korrekte valg. \\

% Rune 20-05-2014 09:35
\textbf{Konklusion}\\
I konklusionen foretages der en opsamling af gruppens endelige produkt og en belysning på, om hvorvidt produktet løser de opstillede kravspecifikationer. \\

\textbf{Perspektivering}\\
Perspektiveringen opstiller interessante udvidelses-potentialer og om hvordan, at programmet kan blive tilpasset, så det kan bruges til andre klubber eller foreninger end blot Aalborg Roklub.



\clearpage
\fi

% Kildeliste
\bibpunct[, ]{[}{]}{;}{n}{,}{,} 		% Definerer de 6 parametre ved Harvard henvisning (bl.a. parantestype og seperatortegn)

% Udseende af litteraturlisten
%\bibliographystyle{Bibtex/Vancouver} 
\bibliographystyle{unsrtnat}            % Inkluderer bl.a. isbn numre

% Oprettelse af links inde i pdf dokumentet
\hypersetup{
    pdftitle={\rapportnavn},
    pdfauthor={\gruppen},
    pdfsubject={\rapportnavn},
    bookmarksnumbered=true,
    bookmarksopen=false,
    bookmarksopenlevel=1,
    colorlinks=false,
    pdfstartview=Fit,
    pdfpagemode=UseOutlines,
    pdfpagelayout=TwoPageRight,
    pdfborder = {0 0 0}
}

% Badness underfull og overfull
\hbadness=10001                         % Slår alle underfull badness warnings fra
\hfuzz=100.002pt                        % Slår de ligegyldige overfull warnings fra

% Figurer
\newcommand{\figur}[5][0]{
		\begin{figure}[H] \centering \em %H / h!
			\includegraphics[width=#5\textwidth, angle=#1]{#2}
			\caption{#3}\label{fig:#4}
		\end{figure}
}

% Environments
\input{Preamble/Environments.tex}